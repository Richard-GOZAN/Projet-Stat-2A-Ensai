% Options for packages loaded elsewhere
\PassOptionsToPackage{unicode}{hyperref}
\PassOptionsToPackage{hyphens}{url}
%
\documentclass[
]{article}
\usepackage{amsmath,amssymb}
\usepackage{iftex}
\ifPDFTeX
  \usepackage[T1]{fontenc}
  \usepackage[utf8]{inputenc}
  \usepackage{textcomp} % provide euro and other symbols
\else % if luatex or xetex
  \usepackage{unicode-math} % this also loads fontspec
  \defaultfontfeatures{Scale=MatchLowercase}
  \defaultfontfeatures[\rmfamily]{Ligatures=TeX,Scale=1}
\fi
\usepackage{lmodern}
\ifPDFTeX\else
  % xetex/luatex font selection
\fi
% Use upquote if available, for straight quotes in verbatim environments
\IfFileExists{upquote.sty}{\usepackage{upquote}}{}
\IfFileExists{microtype.sty}{% use microtype if available
  \usepackage[]{microtype}
  \UseMicrotypeSet[protrusion]{basicmath} % disable protrusion for tt fonts
}{}
\makeatletter
\@ifundefined{KOMAClassName}{% if non-KOMA class
  \IfFileExists{parskip.sty}{%
    \usepackage{parskip}
  }{% else
    \setlength{\parindent}{0pt}
    \setlength{\parskip}{6pt plus 2pt minus 1pt}}
}{% if KOMA class
  \KOMAoptions{parskip=half}}
\makeatother
\usepackage{xcolor}
\usepackage[margin=1in]{geometry}
\usepackage{graphicx}
\makeatletter
\def\maxwidth{\ifdim\Gin@nat@width>\linewidth\linewidth\else\Gin@nat@width\fi}
\def\maxheight{\ifdim\Gin@nat@height>\textheight\textheight\else\Gin@nat@height\fi}
\makeatother
% Scale images if necessary, so that they will not overflow the page
% margins by default, and it is still possible to overwrite the defaults
% using explicit options in \includegraphics[width, height, ...]{}
\setkeys{Gin}{width=\maxwidth,height=\maxheight,keepaspectratio}
% Set default figure placement to htbp
\makeatletter
\def\fps@figure{htbp}
\makeatother
\setlength{\emergencystretch}{3em} % prevent overfull lines
\providecommand{\tightlist}{%
  \setlength{\itemsep}{0pt}\setlength{\parskip}{0pt}}
\setcounter{secnumdepth}{-\maxdimen} % remove section numbering
\ifLuaTeX
  \usepackage{selnolig}  % disable illegal ligatures
\fi
\usepackage{bookmark}
\IfFileExists{xurl.sty}{\usepackage{xurl}}{} % add URL line breaks if available
\urlstyle{same}
\hypersetup{
  hidelinks,
  pdfcreator={LaTeX via pandoc}}

\author{}
\date{\vspace{-2.5em}}

\begin{document}

\section{Introduction}\label{introduction}

La répartition géographique des besoins en soins de santé est un enjeu
majeur pour les politiques publiques, notamment en ce qui concerne
l'accès aux services de médecine de ville. Les inégalités territoriales
dans l'offre et la demande de soins peuvent entraîner des disparités
significatives en matière de santé, affectant particulièrement les
populations vivant dans des zones sous-dotées en professionnels de
santé. Comprendre ces dynamiques spatiales et socio-démographiques est
essentiel pour identifier les zones prioritaires et orienter les
décisions en matière d'allocation des ressources.

Dans ce contexte, ce travail propose une modélisation du nombre de
consultations en médecine de ville à l'échelle communale, en tenant
compte des caractéristiques démographiques, socio-économiques et
spatiales des communes. L'objectif est double : d'une part, analyser les
facteurs influençant la demande de soins, et d'autre part, identifier
les zones susceptibles de dépasser un seuil critique de ``désert
médical''. Pour ce faire, nous nous appuierons sur une base de données
riche et variée, comprenant des informations issues du Système National
des Données de Santé (SNDS) pour la période 2018-2022, ainsi que des
indicateurs socio-démographiques et géographiques.

Notre approche méthodologique repose sur une combinaison de techniques
statistiques et spatiales. Nous commencerons par une analyse descriptive
et cartographique des données pour visualiser les tendances et les
disparités territoriales. Ensuite, nous utiliserons des modèles de
régression de Poisson pour modéliser le nombre de consultations
annuelles, en tenant compte des effets fixes (caractéristiques des
communes) et des effets aléatoires (variations spatiales). Enfin, une
régression logistique binaire sera employée pour évaluer la probabilité
qu'une commune dépasse un seuil prédéfini de ``désert médical''.

Ce travail s'inscrit dans une perspective à la fois académique et
opérationnelle. Sur le plan académique, il contribue à l'étude des
inégalités territoriales en santé en proposant une méthodologie robuste
pour l'analyse spatiale des données de soins. Sur le plan opérationnel,
il fournit des outils pour identifier les zones prioritaires et soutenir
la prise de décision en matière de politiques de santé publique.

\end{document}
