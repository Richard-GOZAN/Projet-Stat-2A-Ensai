% Options for packages loaded elsewhere
\PassOptionsToPackage{unicode}{hyperref}
\PassOptionsToPackage{hyphens}{url}
%
\documentclass[
]{article}
\usepackage{amsmath,amssymb}
\usepackage{iftex}
\ifPDFTeX
  \usepackage[T1]{fontenc}
  \usepackage[utf8]{inputenc}
  \usepackage{textcomp} % provide euro and other symbols
\else % if luatex or xetex
  \usepackage{unicode-math} % this also loads fontspec
  \defaultfontfeatures{Scale=MatchLowercase}
  \defaultfontfeatures[\rmfamily]{Ligatures=TeX,Scale=1}
\fi
\usepackage{lmodern}
\ifPDFTeX\else
  % xetex/luatex font selection
\fi
% Use upquote if available, for straight quotes in verbatim environments
\IfFileExists{upquote.sty}{\usepackage{upquote}}{}
\IfFileExists{microtype.sty}{% use microtype if available
  \usepackage[]{microtype}
  \UseMicrotypeSet[protrusion]{basicmath} % disable protrusion for tt fonts
}{}
\makeatletter
\@ifundefined{KOMAClassName}{% if non-KOMA class
  \IfFileExists{parskip.sty}{%
    \usepackage{parskip}
  }{% else
    \setlength{\parindent}{0pt}
    \setlength{\parskip}{6pt plus 2pt minus 1pt}}
}{% if KOMA class
  \KOMAoptions{parskip=half}}
\makeatother
\usepackage{xcolor}
\usepackage[margin=1in]{geometry}
\usepackage{graphicx}
\makeatletter
\def\maxwidth{\ifdim\Gin@nat@width>\linewidth\linewidth\else\Gin@nat@width\fi}
\def\maxheight{\ifdim\Gin@nat@height>\textheight\textheight\else\Gin@nat@height\fi}
\makeatother
% Scale images if necessary, so that they will not overflow the page
% margins by default, and it is still possible to overwrite the defaults
% using explicit options in \includegraphics[width, height, ...]{}
\setkeys{Gin}{width=\maxwidth,height=\maxheight,keepaspectratio}
% Set default figure placement to htbp
\makeatletter
\def\fps@figure{htbp}
\makeatother
\setlength{\emergencystretch}{3em} % prevent overfull lines
\providecommand{\tightlist}{%
  \setlength{\itemsep}{0pt}\setlength{\parskip}{0pt}}
\setcounter{secnumdepth}{-\maxdimen} % remove section numbering
\ifLuaTeX
  \usepackage{selnolig}  % disable illegal ligatures
\fi
\usepackage{bookmark}
\IfFileExists{xurl.sty}{\usepackage{xurl}}{} % add URL line breaks if available
\urlstyle{same}
\hypersetup{
  hidelinks,
  pdfcreator={LaTeX via pandoc}}

\author{}
\date{\vspace{-2.5em}}

\begin{document}

\section{Présentation du contexte}\label{pruxe9sentation-du-contexte}

\subsection{Intérêt de l'étude}\label{intuxe9ruxeat-de-luxe9tude}

\subsection{Cadre conceptuel de
l'étude}\label{cadre-conceptuel-de-luxe9tude}

Dans cette partie, nous allons définir certainses notions clés qui
apparaissent dans notre étude entre autres, le nombre de visites, le
nombre de visites espérés ainsi que le taux de visite.

\begin{enumerate}
\def\labelenumi{\arabic{enumi}.}
\tightlist
\item
  Nombre de visites espérés
\end{enumerate}

Le nombre de visite espérés, terme qui apparaitra dans notre
modélisation est le nombre de visite qu'il y aurait eu dans chaque
commune si le taux de visite était le même dans dans ces dernières. En
d'autres terme si \(r\) est le taux moyen de visite alors, le nombre de
visite escompté noté \(\mu\) est calculé par :

\[ \mu_i = r * P_i\]

où \(P_i\) est le nombre d'habitants dans la commune \(i\)

\begin{enumerate}
\def\labelenumi{\arabic{enumi}.}
\setcounter{enumi}{1}
\tightlist
\item
  Taux de visite
\end{enumerate}

Le taux de visite n'est rien d'autre que le nombre moyen de visite dans
chaque commune. IL est calculé en divisant le nombre de visite par la
population de la commune en question. En d'autres termes, il s'agit du
nombre de visites que chaque habitant de la commune a effectué en
moyenne.

\[\tau_i = \frac{n_i}{P_i}\]

où \(\tau_i\) et \(n_i\) sont respectivement le taux et le nombre de
visite de la commune \(i\).

\subsection{Revue de littérature}\label{revue-de-littuxe9rature}

La modélisation des visites dans les hôpitaux est cruciale pour
comprendre les dynamiques de la santé publique et pour optimiser la
gestion des ressources médicales. Dans ce contexte, les modèles de
régression de type Poisson généraux, en particulier les modèles mixtes
généralisés (GLMM), sont souvent utilisés pour gérer les données de
comptage, en tenant compte des spécificités démographiques et
géographiques des agglomérations, ici définies par les communes.

Les recherches antérieures, comme celles de Mohebbi et al.~(2011),
montrent l'importance d'intégrer des effets spatiaux dans les modèles de
comptage, surtout lorsqu'on évalue des phénomènes tels que le cancer
œsophagien (EC) dans des provinces spécifiques. Leur étude souligne
comment la non-prise en compte de l'autocorrélation spatiale peut
conduire à une estimation biaisée des effets des variables
socio-économiques sur l'incidence des maladies. Ce phénomène pourrait
également s'appliquer à la modélisation des visites à l'hôpital, où
certains facteurs ouent un rôle significatif.

Selon l'article, trois structures d'autocorrélation peuvent être
appliquées dans le cadre de la régression Poisson lorsqu'on traite des
données de comptage. Ces modèles comprennent :

\begin{enumerate}
\def\labelenumi{\arabic{enumi}.}
\item
  \textbf{Modèle Poisson avec effets aléatoires non spatiaux} : Bien
  qu'utiles, ces modèles négligent l'autocorrélation spatiale, ce qui
  peut conduire à une sous-estimation des erreurs standard.
\item
  \textbf{Modèle avec effets aléatoires spatiaux basés sur la distance}
  : Ce modèle prend en compte l'influence des agglomérations voisines,
  mais peut ne pas capturer efficacement l'hétérogénéité locale.
\item
  \textbf{Modèle avec effets aléatoires de type voisinage} : Utilisant
  la structure conditionnelle autorégressive (CAR), ce modèle est le
  plus adapté pour les données spatiales en intégrant les interactions
  entre les communes. Pour notre étude, le modèle avec effets aléatoires
  de type voisinage est recommandé pour la modélisation du nombre de
  visites à l'hôpital en France, car il permet de mieux prendre en
  compte l'effet de proximité entre communes.
\end{enumerate}

L'application des principes issus de l'article de Mohebbi et al.~peut
enrichir notre étude sur la fréquentation des hôpitaux en France. En
intégrant des effets aléatoires spatiaux adaptés à la structure des
données, nous pouvons réaliser des estimations plus précises et utiles
pour la planification hospitalière.

En ce qui concerne les facteurs explicatifs du nombre de visite dans les
hopitaux, plusieurs études se sont penchées sur ce sujet. De nombreuses
études ont montré que le nombre de consultations médicales est influencé
par divers facteurs sociodémographiques, allant des caractéristiques
individuelles aux contextes socio-économiques et territoriaux.

\begin{enumerate}
\def\labelenumi{\arabic{enumi}.}
\tightlist
\item
  Influence de l'âge et du sexe
\end{enumerate}

L'âge constitue un déterminant majeur du recours aux soins. Les
personnes âgées, en particulier celles de 65 à 79 ans, consultent plus
fréquemment en raison de la prévalence accrue de maladies chroniques et
du suivi médical nécessaire à leur prise en charge {[}@statcan2022{]}.
En revanche, la population jeune et en bonne santé présente une
utilisation plus sporadique des services médicaux. Le sexe est également
un facteur différenciant important. De manière générale, les femmes
consultent plus fréquemment que les hommes. Cette différence est
attribuée aux besoins spécifiques en santé reproductive, mais aussi à
une plus grande propension à rechercher des soins préventifs
{[}@bag2024{]}. En revanche, les hommes, notamment dans les catégories
socio-professionnelles les plus actives, ont tendance à sous-utiliser
les services de soins, ce qui peut entraîner des diagnostics plus
tardifs et des complications médicales accrues.

\begin{enumerate}
\def\labelenumi{\arabic{enumi}.}
\setcounter{enumi}{1}
\tightlist
\item
  Impact du statut socio-économique et du niveau d'éducation
\end{enumerate}

Le revenu et la précarité économique influencent considérablement
l'accès aux soins. Les individus à revenu élevé bénéficient généralement
d'un meilleur accès aux consultations médicales, notamment grâce à une
plus grande couverture sociale et des assurances complémentaires leur
permettant de réduire les coûts associés aux soins {[}@bvs2023{]}. À
l'inverse, les personnes en situation de précarité rencontrent des
obstacles financiers, administratifs et culturels qui limitent leur
recours aux soins, malgré des besoins souvent accrus en raison de
conditions de vie plus précaires. Le niveau d'éducation joue un rôle clé
dans la fréquentation des services de santé. Une instruction plus élevée
est associée à une meilleure connaissance des risques sanitaires et à
une adoption plus proactive des comportements de prévention, ce qui
entraîne un recours plus fréquent aux soins médicaux {[}@bvs2023{]}. En
revanche, un faible niveau d'éducation est souvent corrélé à un moindre
suivi médical et à une utilisation plus tardive des services de soins,
notamment en cas de complications.

\begin{enumerate}
\def\labelenumi{\arabic{enumi}.}
\setcounter{enumi}{2}
\tightlist
\item
  Influence du contexte familial et de l'environnement social
\end{enumerate}

L'état matrimonial influence également la fréquence des consultations
médicales. Les personnes mariées ou vivant en couple consultent
davantage, bénéficiant du soutien d'un conjoint qui peut inciter à
prendre soin de sa santé et à consulter régulièrement un médecin
{[}@statcan2022{]}. Par ailleurs, l'accès à un médecin traitant ou de
famille constitue un déterminant important. Les patients disposant d'un
suivi médical régulier sont plus enclins à effectuer des consultations
préventives et à être orientés rapidement vers des spécialistes si
nécessaire {[}@statcan2022{]}.

\begin{enumerate}
\def\labelenumi{\arabic{enumi}.}
\setcounter{enumi}{3}
\tightlist
\item
  Perception de l'état de santé et accès géographique aux soins
\end{enumerate}

La perception de la santé est un facteur déterminant du recours aux
soins. Les individus qui considèrent leur état de santé comme excellent
ou très bon consultent rarement, tandis que ceux qui ont une perception
négative de leur état de santé ont tendance à multiplier les visites
médicales {[}@statcan2022{]}. Enfin, les inégalités spatiales dans
l'accès aux soins modulent également la fréquence des consultations. En
milieu urbain, la densité médicale plus élevée facilite l'accès aux
services de soins, tandis qu'en zones rurales ou médicalement
sous-dotées, les délais d'attente et les distances à parcourir
constituent des freins majeurs {[}@irdes2020{]}.

\end{document}
