% Options for packages loaded elsewhere
\PassOptionsToPackage{unicode}{hyperref}
\PassOptionsToPackage{hyphens}{url}
%
\documentclass[
]{article}
\usepackage{amsmath,amssymb}
\usepackage{iftex}
\ifPDFTeX
  \usepackage[T1]{fontenc}
  \usepackage[utf8]{inputenc}
  \usepackage{textcomp} % provide euro and other symbols
\else % if luatex or xetex
  \usepackage{unicode-math} % this also loads fontspec
  \defaultfontfeatures{Scale=MatchLowercase}
  \defaultfontfeatures[\rmfamily]{Ligatures=TeX,Scale=1}
\fi
\usepackage{lmodern}
\ifPDFTeX\else
  % xetex/luatex font selection
\fi
% Use upquote if available, for straight quotes in verbatim environments
\IfFileExists{upquote.sty}{\usepackage{upquote}}{}
\IfFileExists{microtype.sty}{% use microtype if available
  \usepackage[]{microtype}
  \UseMicrotypeSet[protrusion]{basicmath} % disable protrusion for tt fonts
}{}
\makeatletter
\@ifundefined{KOMAClassName}{% if non-KOMA class
  \IfFileExists{parskip.sty}{%
    \usepackage{parskip}
  }{% else
    \setlength{\parindent}{0pt}
    \setlength{\parskip}{6pt plus 2pt minus 1pt}}
}{% if KOMA class
  \KOMAoptions{parskip=half}}
\makeatother
\usepackage{xcolor}
\usepackage[margin=1in]{geometry}
\usepackage{graphicx}
\makeatletter
\def\maxwidth{\ifdim\Gin@nat@width>\linewidth\linewidth\else\Gin@nat@width\fi}
\def\maxheight{\ifdim\Gin@nat@height>\textheight\textheight\else\Gin@nat@height\fi}
\makeatother
% Scale images if necessary, so that they will not overflow the page
% margins by default, and it is still possible to overwrite the defaults
% using explicit options in \includegraphics[width, height, ...]{}
\setkeys{Gin}{width=\maxwidth,height=\maxheight,keepaspectratio}
% Set default figure placement to htbp
\makeatletter
\def\fps@figure{htbp}
\makeatother
\setlength{\emergencystretch}{3em} % prevent overfull lines
\providecommand{\tightlist}{%
  \setlength{\itemsep}{0pt}\setlength{\parskip}{0pt}}
\setcounter{secnumdepth}{-\maxdimen} % remove section numbering
% definitions for citeproc citations
\NewDocumentCommand\citeproctext{}{}
\NewDocumentCommand\citeproc{mm}{%
  \begingroup\def\citeproctext{#2}\cite{#1}\endgroup}
\makeatletter
 % allow citations to break across lines
 \let\@cite@ofmt\@firstofone
 % avoid brackets around text for \cite:
 \def\@biblabel#1{}
 \def\@cite#1#2{{#1\if@tempswa , #2\fi}}
\makeatother
\newlength{\cslhangindent}
\setlength{\cslhangindent}{1.5em}
\newlength{\csllabelwidth}
\setlength{\csllabelwidth}{3em}
\newenvironment{CSLReferences}[2] % #1 hanging-indent, #2 entry-spacing
 {\begin{list}{}{%
  \setlength{\itemindent}{0pt}
  \setlength{\leftmargin}{0pt}
  \setlength{\parsep}{0pt}
  % turn on hanging indent if param 1 is 1
  \ifodd #1
   \setlength{\leftmargin}{\cslhangindent}
   \setlength{\itemindent}{-1\cslhangindent}
  \fi
  % set entry spacing
  \setlength{\itemsep}{#2\baselineskip}}}
 {\end{list}}
\usepackage{calc}
\newcommand{\CSLBlock}[1]{\hfill\break\parbox[t]{\linewidth}{\strut\ignorespaces#1\strut}}
\newcommand{\CSLLeftMargin}[1]{\parbox[t]{\csllabelwidth}{\strut#1\strut}}
\newcommand{\CSLRightInline}[1]{\parbox[t]{\linewidth - \csllabelwidth}{\strut#1\strut}}
\newcommand{\CSLIndent}[1]{\hspace{\cslhangindent}#1}
\ifLuaTeX
  \usepackage{selnolig}  % disable illegal ligatures
\fi
\usepackage{bookmark}
\IfFileExists{xurl.sty}{\usepackage{xurl}}{} % add URL line breaks if available
\urlstyle{same}
\hypersetup{
  hidelinks,
  pdfcreator={LaTeX via pandoc}}

\author{}
\date{\vspace{-2.5em}}

\begin{document}

\section{Présentation du contexte}\label{pruxe9sentation-du-contexte}

\subsection{Cadre conceptuel de
l'étude}\label{cadre-conceptuel-de-luxe9tude}

Dans cette partie, nous allons définir certainses notions clés qui
apparaissent dans notre étude : le taux de visite et la notion de
voisin.

\subsubsection{Taux de visite}\label{taux-de-visite}

Le taux de visite n'est rien d'autre que le nombre moyen de visite dans
chaque commune. IL est calculé en divisant le nombre de visite par la
population de la commune en question. En d'autres termes, il s'agit du
nombre de visites que chaque habitant de la commune a effectué en
moyenne.

\[\tau_i = \frac{n_i}{P_i}\]

où \(\tau_i\) et \(n_i\) sont respectivement le taux et le nombre de
visite de la commune \(i\).

\subsubsection{La notion de voisin}\label{la-notion-de-voisin}

Il est indispensable de définir le voisinage d'un objet spatial pour
quantifier l'influence réciproque entre entités. La définition formelle
se traduit par la détermination d'un ensemble de couples \((i,j)\)
d'objets spatiaux, avec la contrainte qu'un objet ne peut être voisin de
lui-même, c'est-à-dire que pour tout \(i\), \((i,i)\) n'est pas
considéré.

Une méthode classique consiste à construire une matrice de voisinage
binaire \(W\) définie par : \[
W_{ij} = \begin{cases} 
1, & \text{si } i \text{ et } j \text{ sont considérés comme voisins} \\
0, & \text{sinon}
\end{cases}
\] Les méthodes pour définir les voisins sont diverses :

\begin{itemize}
  \item \textbf{Basée sur la distance :}  
  On utilise par exemple la distance euclidienne,
  $$
  d(i,j) = \sqrt{(x_i - x_j)^2 + (y_i - y_j)^2},
  $$
pour définir deux objets comme voisins si \(d(i,j)\) est inférieure à un seuil prédéfini, ou en pondérant cette distance par une fonction décroissante. Dans le cadre de cette étude, nous utiliserons la distance de Haversine, compte tenu des données utilisées (longitude et latitude)

\item \textbf{Basée sur la contiguïté :}  
  Pour des données surfaciques (comme des zones administratives), les voisins sont définis en fonction du partage d'une frontière commune. On distingue par exemple la contiguïté \emph{Rook} (deux zones sont voisines si elles partagent un segment de frontière) et la contiguïté \emph{Queen} (elles sont voisines si elles partagent au moins un point).
  
  \item \textbf{Basée sur l'optimisation d'une trajectoire :}  
  Une autre approche consiste à ordonner les points selon un chemin optimisé (par exemple, le plus court chemin ou le cycle hamiltonien dans le problème du voyageur de commerce). Les voisins d'un point sont alors définis comme les points qui se suivent immédiatement dans cet ordre.
\end{itemize}

\subsubsection{Distance de Haversine}\label{distance-de-haversine}

La distance de Haversine est une mesure de la distance entre deux points
sur une sphère, basée sur leurs coordonnées géographiques (\(latitude\)
et \(longitude\)). Elle est particulièrement utile pour les données
géographiques projetées sur une surface sphérique, comme la Terre.

Si l'on considère deux points (\(i\)) et (\(j\)), la distance
(\(d_{ij}\)) entre ces deux points sur la surface d'une sphère de rayon
(\(r\)) est donnée par :

\[
 d_{ij} = 2r \cdot \arcsin\left(\sqrt{\sin^2\left(\frac{\phi_j - \phi_i}{2}\right) + \cos(\phi_i)\cos(\phi_j)\sin^2\left(\frac{\lambda_j - \lambda_i}{2}\right)}\right)
\]

Où :

\begin{itemize}
\item
  \(r\) : Rayon de la Terre (environ 6371 km).
\item
  \(\phi_i, \phi_j\) : Latitudes des points \(i\) et \(j\) (en radians).
\item
  \(\lambda_i, \lambda_j\) : Longitudes des points \(i\) et \(j\) (en
  radians). Après calcul nous avons ces statistiques sur nos distances.
\end{itemize}

\subsection{Revue de littérature}\label{revue-de-littuxe9rature}

La modélisation des visites dans les hôpitaux est cruciale pour
comprendre les dynamiques de la santé publique et optimiser la gestion
des ressources médicales. En 2023, les services d'urgences en France ont
enregistré environ 20,9 millions de passages, marquant une légère
diminution par rapport à 2022 (Egora 2023). Parallèlement, l'accès aux
consultations médicales spécialisées demeure préoccupant. En 2024, le
délai moyen pour obtenir un rendez-vous chez un dermatologue était de 36
jours, l'un des plus longs parmi les spécialités médicales (BFMTV 2024).
Ces difficultés d'accès aux soins ont conduit un nombre croissant de
patients à se tourner vers les services d'urgences pour des
consultations non urgentes, accentuant ainsi la saturation de ces
services. Dans ce contexte, il est crucial de comprendre les facteurs
qui influencent le nombre de consultations médicales, afin d'allouer les
ressources de manière optimale sur le plan matériel, humain et
financier, tout en prévenant des situations de surcharge des
établissements de santé.

De nombreuses études ont montré que le nombre de consultations médicales
est influencé par une multitude de facteurs sociodémographiques, allant
des caractéristiques individuelles aux contextes socio-économiques et
territoriaux. L'âge est un déterminant majeur du recours aux soins. Les
personnes âgées, en particulier celles de 65 à 79 ans, consultent plus
fréquemment en raison de la prévalence accrue de maladies chroniques et
du suivi médical nécessaire à leur prise en charge. En revanche, les
jeunes adultes, en bonne santé, présentent une utilisation plus
sporadique des services médicaux.

Le sexe constitue également un facteur différenciant : les femmes
consultent plus fréquemment que les hommes, en raison de besoins
spécifiques en santé reproductive et d'une plus grande propension à
rechercher des soins préventifs. À l'inverse, les hommes, notamment dans
les catégories socio-professionnelles les plus actives, tendent à
sous-utiliser les services de soins, ce qui peut entraîner des
diagnostics plus tardifs et des complications médicales accrues.

Le statut socio-économique et le niveau d'éducation influencent
également de manière significative l'accès aux soins. Les individus à
revenu élevé bénéficient généralement d'un meilleur accès aux
consultations médicales, grâce à une couverture sociale plus complète et
des assurances complémentaires qui allègent les coûts des soins. À
l'inverse, les personnes en situation de précarité rencontrent des
obstacles financiers, administratifs et culturels qui limitent leur
recours aux soins, malgré des besoins souvent accrus en raison de
conditions de vie plus précaires.

Le niveau d'éducation joue un rôle clé dans la fréquentation des
services de santé. Une meilleure instruction est associée à une
meilleure connaissance des risques sanitaires et à une adoption plus
proactive des comportements de prévention, entraînant un recours plus
fréquent aux soins médicaux. À l'inverse, un faible niveau d'éducation
est souvent corrélé à un moindre suivi médical et à une utilisation plus
tardive des services de soins, notamment en cas de complications.

La perception de la santé et l'accès géographique aux soins sont
d'autres facteurs déterminants. Les individus qui jugent leur état de
santé comme étant excellent ou très bon consultent moins fréquemment,
tandis que ceux ayant une perception négative de leur état de santé sont
plus enclins à multiplier les visites médicales (Canada 2022). En ce qui
concerne l'accès géographique, les inégalités spatiales jouent un rôle
clé dans la fréquence des consultations. En milieu urbain, la densité
médicale plus élevée facilite l'accès aux soins, tandis qu'en zones
rurales ou médicalement sous-dotées, les délais d'attente et la distance
à parcourir constituent des freins majeurs à l'accès aux soins (Irdes
2020).

L'étude de Nkoua Mbon et al.~(2021) sur les facteurs associés au faible
niveau de fréquentation du centre de santé intégré de Pondila identifie
plusieurs obstacles à la fréquentation, notamment le coût des soins, le
mauvais état des routes, le chômage, le mauvais accueil par le personnel
soignant et la non-disponibilité de médicaments. Ces facteurs soulignent
la complexité de l'accès aux soins, notamment dans les zones plus
rurales ou moins développées.

L'étude de Julie Dumouchel (2012) sur les pratiques des généralistes
normands face aux urgences médicales met en évidence plusieurs éléments
clés. Les facteurs organisationnels tels que la disponibilité des
structures de soins, les horaires d'ouverture des cabinets et la
collaboration avec d'autres professionnels de santé sont essentiels dans
la gestion des urgences. Les facteurs personnels, notamment l'expérience
professionnelle, la formation continue et la confiance en soi des
médecins, jouent également un rôle crucial dans la prise en charge
efficace des situations urgentes.

En outre, la présentation des patients, leur niveau d'urgence perçu et
leurs attentes influencent la façon dont les médecins abordent les cas
urgents. L'accès aux équipements médicaux, la qualité des
infrastructures et la disponibilité des services d'urgence jouent aussi
un rôle important. Cette étude met en lumière la complexité de la
gestion des urgences en médecine générale, influencée par une série de
facteurs interconnectés, et souligne la nécessité d'améliorer
l'organisation des soins, de renforcer la formation des médecins et
d'optimiser les ressources disponibles pour une gestion plus efficace
des urgences médicales en Normandie.

L'un des facteurs pouvant influencer le nombre de consultations dans une
zone est le renoncement aux soins En effet, selon le rapport du DREES
(Renoncement aux soins : la faible densité médicale est un facteur
aggravant pour les personnes pauvres), la question du renoncement aux
soins dans les milieux urbains est influencée par une multitude de
facteurs socio-économiques et structurels. Selon l'enquête Statistiques
sur les ressources et conditions de vie (SRCV), une proportion
significative de la population en France (3,1 \%) renonce à des soins
médicaux, un phénomène particulièrement prononcé chez les personnes
pauvres qui vivent dans des zones à faible densité médicale. Ces
individus sont jusqu'à huit fois plus susceptibles de renoncer aux soins
en raison de l'accessibilité limitée à des médecins généralistes.

\phantomsection\label{refs}
\begin{CSLReferences}{1}{0}
\bibitem[\citeproctext]{ref-bfmtv2024}
BFMTV. 2024. {``Généraliste, Dermatologue, Pédiatre : Quel Est Le Délai
Pour Obtenir Un Rendez-Vous Chez Le Médecin En 2024 ?''}
\url{https://www.bfmtv.com/tech/vie-numerique/sur-doctolib-la-moitie-des-consultations-de-generalistes-sont-obtenues-en-moins-de-3-jours_AD-202404230604.html}.

\bibitem[\citeproctext]{ref-statcan2022}
Canada, Statistique. 2022. {``Fréquence Des Consultations Médicales Et
Facteurs Sociodémographiques.''} \url{https://www150.statcan.gc.ca}.

\bibitem[\citeproctext]{ref-egora2023}
Egora. 2023. {``Hôpital : La Part d'ambulatoire Augmente de 6,5 \% , Les
Séjours Diminuent.''}
\url{https://www.egora.fr/actus-pro/hopitaux/hopital-la-part-dambulatoire-augmente-de-65-les-sejours-diminuent}.

\bibitem[\citeproctext]{ref-irdes2020}
Irdes. 2020. {``Inégalités Spatiales d'accessibilité Aux Soins
Médicaux.''} \url{https://www.irdes.fr}.

\end{CSLReferences}

\end{document}
