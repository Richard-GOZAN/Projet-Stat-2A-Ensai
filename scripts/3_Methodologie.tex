\section{Méthodologie}\label{muxe9thodologie}

\subsection{Présentation des
données}\label{pruxe9sentation-des-donnuxe9es}

Les données que nous avons utilisées nous proviennent de \ldots{}

\subsection{Motivation}\label{motivation}

Les modèles linéaires généralisés à effets mixtes (GLMM) combinent :

\begin{itemize}
\item
  Les caractéristiques des modèles linéaires généralisés (GLM) pour
  modéliser des variables non-normalement distribuées.
\item
  Les propriétés des modèles à effets mixtes pour gérer des données
  groupées ou hiérarchiques.
\end{itemize}

\subsection{Modèles Linéaires
Généralisés}\label{moduxe8les-linuxe9aires-guxe9nuxe9ralisuxe9s}

Un GLM relie le prédicteur linéaire \(\eta\) à la moyenne \(\mu\) de la
réponse à travers une fonction de lien \(g\) :

\[ g(\mu) = \eta = \beta_0 + \sum_{i=1}^m \beta_i x_i \]

Les distributions possibles incluent :

\begin{itemize}
\item
  \textbf{Normale} : Régression linéaire classique, avec lien identité.
\item
  \textbf{Binomiale} : Régression logistique pour données binaires, avec
  lien logit.
\item
  \textbf{Poisson} : Régression de Poisson pour données de comptage,
  avec lien logarithmique.
\end{itemize}

\subsubsection{Régression Logistique}\label{ruxe9gression-logistique}

Modélise une réponse binaire (\(y \sim B(n, p)\)), où \(p\) est la
probabilité de succès :

\[ P(y | n, p) = \binom{n}{y} p^y (1-p)^{n-y} \]

La probabilité \(p\) est reliée au prédicteur par la fonction logistique
:

\[ p = \frac{1}{1 + e^{-\eta}} \quad \text{où} \quad \eta = \beta_0 + \sum_{i=1}^m \beta_i x_i. \]

Le log-vraisemblance est exprimé comme :

\[ \ell(\boldsymbol{\beta}) = \sum_{i=1}^n \left[ y_i \log{p_i} + (1-y_i) \log{(1-p_i)} \right] \]

\subsubsection{Régression de Poisson}\label{ruxe9gression-de-poisson}

Utilisée pour modéliser des données de comptage
(\(y \sim Pois(\lambda)\)), où \(\lambda\) est la moyenne et la variance
:

\[ P(y | \lambda) = \frac{\lambda^y}{y!} e^{-\lambda} \]

Le lien logarithmique assure \(\lambda > 0\) :

\[ \log{\lambda} = \beta_0 + \sum_{i=1}^m \beta_i x_i \]

L'espérance est \(E[y] = \lambda\).

\subsection{Modèles Linéaires
Mixtes}\label{moduxe8les-linuxe9aires-mixtes}

Ces modèles ajoutent des termes d'effets aléatoires
\(\mathbf{Z} \mathbf{u}\) au prédicteur linéaire :

\[ \mathbf{y} = \mathbf{X} \boldsymbol{\beta} + \mathbf{Z} \mathbf{u} + \boldsymbol{\varepsilon}, \]
avec :

\begin{itemize}
\item
  \(\mathbf{u} \sim N(\mathbf{0}, \mathbf{G})\), les effets aléatoires.
\item
  \(\boldsymbol{\varepsilon} \sim N(\mathbf{0}, \mathbf{R})\), les
  résidus.
\end{itemize}

La matrice de covariance totale est :

\[ \mathrm{Var}(\mathbf{y}) = \mathbf{Z} \mathbf{G} \mathbf{Z}^T + \mathbf{R}. \]

Les paramètres sont estimés par maximum de vraisemblance (ML) ou par
vraisemblance restreinte (REML).

\subsection{Modèles Linéaires Généralisés à Effets Mixtes
(GLMM)}\label{moduxe8les-linuxe9aires-guxe9nuxe9ralisuxe9s-uxe0-effets-mixtes-glmm}

Un GLMM étend les GLM en intégrant des effets aléatoires :

\[ g(\mu) = \mathbf{X} \boldsymbol{\beta} + \mathbf{Z} \mathbf{u}, \] où
:

\begin{itemize}
\item
  \(g(\cdot)\) est la fonction de lien.
\item
  \(\mathbf{u} \sim N(\mathbf{0}, \mathbf{G})\) est le vecteur d'effets
  aléatoires.
\end{itemize}

Les paramètres sont estimés via des méthodes comme :

\begin{itemize}
\item
  Approximations Laplaciennes.
\item
  Quadrature gaussienne adaptative.
\item
  Méthodes MCMC (chaînes de Markov Monte Carlo).
\end{itemize}

\subsection{Prédictions et
Simulations}\label{pruxe9dictions-et-simulations}

Les GLMM permettent deux types de prédictions :

\begin{itemize}
\item
  \textbf{Conditionnelles} : Basées sur les effets aléatoires
  spécifiques (\(\mathbf{u}\)).
\item
  \textbf{Marginales} : En intégrant sur les effets aléatoires.
\end{itemize}

Les simulations utilisent des approches paramétriques pour évaluer la
variabilité et tester les hypothèses. Une approche courante est le
bootstrap paramétrique :

\begin{enumerate}
\def\labelenumi{\arabic{enumi}.}
\item
  Générer des données simulées basées sur les paramètres estimés.
\item
  Réajuster le modèle pour chaque jeu de données simulé.
\item
  Analyser la distribution des estimations obtenues.
\end{enumerate}

