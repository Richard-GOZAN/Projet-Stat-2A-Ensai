% Options for packages loaded elsewhere
\PassOptionsToPackage{unicode}{hyperref}
\PassOptionsToPackage{hyphens}{url}
%
\documentclass[
]{article}
\usepackage{amsmath,amssymb}
\usepackage{iftex}
\ifPDFTeX
  \usepackage[T1]{fontenc}
  \usepackage[utf8]{inputenc}
  \usepackage{textcomp} % provide euro and other symbols
\else % if luatex or xetex
  \usepackage{unicode-math} % this also loads fontspec
  \defaultfontfeatures{Scale=MatchLowercase}
  \defaultfontfeatures[\rmfamily]{Ligatures=TeX,Scale=1}
\fi
\usepackage{lmodern}
\ifPDFTeX\else
  % xetex/luatex font selection
\fi
% Use upquote if available, for straight quotes in verbatim environments
\IfFileExists{upquote.sty}{\usepackage{upquote}}{}
\IfFileExists{microtype.sty}{% use microtype if available
  \usepackage[]{microtype}
  \UseMicrotypeSet[protrusion]{basicmath} % disable protrusion for tt fonts
}{}
\makeatletter
\@ifundefined{KOMAClassName}{% if non-KOMA class
  \IfFileExists{parskip.sty}{%
    \usepackage{parskip}
  }{% else
    \setlength{\parindent}{0pt}
    \setlength{\parskip}{6pt plus 2pt minus 1pt}}
}{% if KOMA class
  \KOMAoptions{parskip=half}}
\makeatother
\usepackage{xcolor}
\usepackage[margin=1in]{geometry}
\usepackage{graphicx}
\makeatletter
\def\maxwidth{\ifdim\Gin@nat@width>\linewidth\linewidth\else\Gin@nat@width\fi}
\def\maxheight{\ifdim\Gin@nat@height>\textheight\textheight\else\Gin@nat@height\fi}
\makeatother
% Scale images if necessary, so that they will not overflow the page
% margins by default, and it is still possible to overwrite the defaults
% using explicit options in \includegraphics[width, height, ...]{}
\setkeys{Gin}{width=\maxwidth,height=\maxheight,keepaspectratio}
% Set default figure placement to htbp
\makeatletter
\def\fps@figure{htbp}
\makeatother
\setlength{\emergencystretch}{3em} % prevent overfull lines
\providecommand{\tightlist}{%
  \setlength{\itemsep}{0pt}\setlength{\parskip}{0pt}}
\setcounter{secnumdepth}{-\maxdimen} % remove section numbering
\ifLuaTeX
  \usepackage{selnolig}  % disable illegal ligatures
\fi
\usepackage{bookmark}
\IfFileExists{xurl.sty}{\usepackage{xurl}}{} % add URL line breaks if available
\urlstyle{same}
\hypersetup{
  hidelinks,
  pdfcreator={LaTeX via pandoc}}

\author{}
\date{\vspace{-2.5em}}

\begin{document}

\section{Méthodologie}\label{muxe9thodologie}

\subsection{Source des données}\label{source-des-donnuxe9es}

L'étude repose sur des données issues du Système National des Données de
Santé (SNDS), couvrant la période 2018-2022 et portant sur environ 5000
communes. Ces données permettent d'analyser le nombre de consultations
en médecine de ville en tenant compte des disparités territoriales et
des caractéristiques locales. Nous avons trois bases essentielles. La
première source est une base démographique contenant des données
détaillées sur la répartition de la population par sexe et par tranche
d'âge, ainsi que des indicateurs généraux tels que la population
municipale et la structure des ménages. La deuxième source est une base
généralisée intégrant des informations socio-économiques, notamment sur
les statuts matrimoniaux, les catégories socioprofessionnelles et le
marché de l'emploi. Cette base permet d'étudier la composition sociale
des communes et d'évaluer certains phénomènes tels que le taux
d'activité ou la prévalence des unions libres. Enfin, une troisième base
a été utilisée pour compléter les données géographiques, en particulier
pour renseigner les latitudes et longitudes manquantes de certaines
communes.

Les informations utilisées concernent principalement le volume des
consultations médicales et leur répartition géographique. Des données
complémentaires sur le contexte communal, telles que la densité de
population et l'accessibilité aux soins, permettent d'affiner l'analyse.
L'intégration de ces éléments facilite une approche spatiale de la
modélisation, essentielle pour détecter d'éventuelles inégalités d'accès
aux soins.

L'ensemble des données a été anonymisé et traité conformément aux normes
en vigueur, garantissant ainsi la confidentialité des informations
exploitées. Cette base constitue une ressource précieuse pour mieux
comprendre les dynamiques d'accès aux soins en médecine de ville et
proposer des modèles adaptés aux spécificités territoriales.

\subsection{Traitement réalisés sur la base de
données}\label{traitement-ruxe9alisuxe9s-sur-la-base-de-donnuxe9es}

Après la fusion des bases de données, un ensemble de traitements a été
réalisé afin de structurer les informations et garantir la cohérence des
analyses. La première étape a consisté à nettoyer et harmoniser les noms
des variables pour faciliter leur manipulation. Ce travail a inclus la
suppression des accents, le remplacement des espaces et caractères
spéciaux par des underscores et la conversion en minuscules. De plus,
certaines incohérences ont été corrigées, notamment des erreurs
typographiques dans les intitulés des tranches d'âge.

Une fois cette normalisation effectuée, les différentes bases ont été
fusionnées. Une jointure interne a été réalisée entre les bases
démographiques et socio-économiques en utilisant le code unique des
communes, ce qui a permis de conserver uniquement les communes présentes
dans les deux sources. Ensuite, une jointure gauche avec la base des
coordonnées géographiques a été effectuée pour compléter les
informations manquantes sur la latitude et la longitude des communes
concernées. Lors de cette étape, une attention particulière a été portée
pour éviter la duplication des données et sélectionner les valeurs les
plus pertinentes en cas de conflit entre plusieurs sources.

L'étape suivante a consisté à structurer et créer de nouvelles variables
analytiques. En ce qui concerne la répartition de la population, les
différentes tranches d'âge disponibles ont été regroupées en trois
grandes catégories : 0-24 ans, 25-64 ans et 65 ans et plus. Ce
regroupement permet d'obtenir une vision synthétique de la structure
démographique tout en conservant des distinctions essentielles entre les
jeunes, les actifs et les seniors. Pour chacune de ces nouvelles
catégories, un indicateur de proportion a été calculé en rapportant la
population de chaque groupe à la population totale de la commune.

Un travail spécifique a été réalisé pour enrichir l'analyse
socio-économique. Le pourcentage des personnes en union libre a été
calculé à partir des données disponibles sur les statuts matrimoniaux,
et la part des ouvriers a été estimée en rapportant cette catégorie à la
population totale. Le pourcentage des personnes sans emploi a été évalué
en prenant comme référence la population âgée de 0 à 64 ans, excluant
ainsi les retraités pour obtenir un indicateur plus pertinent sur la
population active. De plus, un indicateur mesurant la proportion des
familles avec trois enfants ou plus a été créé afin d'étudier la
dynamique familiale dans les communes étudiées.

Un des traitements majeurs effectués a été la création d'une variable
inédite : le taux de consultation. Cette variable, qui n'existait pas
initialement dans la base, a été construite en rapportant le nombre
total de consultations enregistrées dans chaque commune à la population
municipale. Cette mesure permet d'évaluer la fréquence des consultations
médicales ou administratives en fonction de la taille de la population
et constitue un indicateur clé pour analyser l'accessibilité et
l'utilisation des services sur le territoire étudié. Une version
spécifique de cet indicateur a également été calculée pour la population
de 19 ans et plus, en excluant les plus jeunes afin de mieux capter les
tendances de consultation chez les adultes.

Enfin, un effort particulier a été porté sur la visualisation des
données, notamment avec la construction d'une pyramide des âges. Pour
cela, les effectifs masculins et féminins ont été extraits et
réorganisés par tranche d'âge afin de respecter la convention des
pyramides démographiques, affichant les populations masculines en
valeurs négatives et les populations féminines en valeurs positives.
Cette représentation permet d'identifier les déséquilibres entre les
classes d'âge et de mieux comprendre la structure démographique des
communes analysées.

Une fois ces traitements finalisés, la base de données enrichie a été
exportée sous un format exploitable, intégrant l'ensemble des nouvelles
variables créées ainsi que les indicateurs de taux de natalité et de
mortalité. Ce travail de préparation assure une qualité optimale des
données et permet de mener des analyses détaillées sur la dynamique
démographique et socio-économique des communes étudiées.

\subsection{Concepts fondamentaux en statistique
spatiale}\label{concepts-fondamentaux-en-statistique-spatiale}

\subsubsection{Autocorrélation
spatiale}\label{autocorruxe9lation-spatiale}

L'autocorrélation spatiale désigne la dépendance statistique entre des
observations géographiquement proches. En d'autres termes, les valeurs
prises par une variable en un lieu donné sont influencées par les
valeurs observées dans les localisations voisines. Cette dépendance peut
être positive, lorsque des valeurs similaires se regroupent, ou
négative, lorsqu'une valeur élevée en un point est associée à une valeur
faible dans les zones environnantes.

Trois (03) formes d'interaction spatiale se dégagent du modèle spatial
de base :

\begin{itemize}
  \item une interaction endogène : la décision économique d’un agent ou d’une zone géographique va dépendre de la décision de ses voisins;
  \item une interaction exogène : la décision économique d’un agent va dépendre des caractéristiques observables de ses voisins;
  \item une corrélation spatiale des effets : liée à de mêmes caractéristiques inobservées entre les agents.
\end{itemize}

\subsubsection{Diagramme de Moran}\label{diagramme-de-moran}

Le diagramme de Moran est un outil permettant d'analyser la structure
spatiale d'une variable. Il représente un nuage de points où :

\begin{itemize}
\tightlist
\item
  L'axe des abscisses affiche les valeurs centrées de la variable
  d'intérêt \(y\).\\
\item
  L'axe des ordonnées affiche les valeurs moyennes de cette variable
  pour les observations voisines \(W y\), où \(W\) est la matrice de
  poids normalisée.
\end{itemize}

\underline{\textbf{Utilité du Diagramme de Moran}} \textbackslash{}

Le diagramme de Moran permet d'identifier les structures spatiales
dominantes en observant la répartition des points dans des quadrants. Il
aide également à détecter les observations atypiques qui s'écartent du
modèle spatial général et à confirmer l'existence d'une autocorrélation
spatiale en complément de l'indice de Moran, qui quantifie cette
relation. Cet outil visuel est ainsi essentiel pour explorer la
dépendance spatiale et comprendre les structures spatiales sous-jacentes
d'une variable d'intérêt.

\subsubsection{Matrice des poids
spatiaux}\label{matrice-des-poids-spatiaux}

Pour quantifier la proximité spatiale entre unités géographiques, on
utilise une matrice de poids spatiaux notée \(W\). Cette matrice
représente les relations de voisinage et permet d'introduire la
structure spatiale dans les modèles économétriques. La matrice de poids
peut être une matrice de contiguïté binaire ou peut tenir compte de la
distance entre les zones géographiques. Cette étude utilise une matrice
de poids basée sur la distance et contient des pondérations inversément
proportionnelles à la distance entre les régions.

\subsubsection{Mesure de la corrélation
spatiale}\label{mesure-de-la-corruxe9lation-spatiale}

\paragraph{Indices de corrélation
spatiale}\label{indices-de-corruxe9lation-spatiale}

L'un des indicateurs les plus utilisés est l'indice de Moran. Il évalue
la similitude des valeurs d'une variable entre différentes entités
géographiques (par exemple, des communes) en fonction de leur proximité
spatiale. Il se base sur la matrice spatiale de poids (\(W\)), qui
définit les relations entre ces entités. Il se calcule comme suit :

\[
I = \frac{N}{\sum_{i} \sum_{j} W_{ij}} \times \frac{\sum_{i} \sum_{j} W_{ij} (y_i - \bar{y})(y_j - \bar{y})}{\sum_{i} (y_i - \bar{y})^2}
\]

où \(y_i\) est la valeur de la variable d'intérêt en un point \(i\),
\(\bar{y}\) est la moyenne empirique de cette variable et \(W_{ij}\)
représente l'élément \((i, j)\) de la matrice de poids.

D'autres indices existent, comme la \textbf{statistique de Geary}, qui
est moins sensible aux valeurs extrêmes, et les
\textbf{indicateurs locaux d'autocorrélation spatiale (LISA)}, qui
permettent d'identifier des clusters spatiaux spécifiques.

\paragraph{Construction de la matrice de
poids}\label{construction-de-la-matrice-de-poids}

\hfill\break
Pour construire la matrice de poids, nous avons suivi ces étapes.\\

\begin{enumerate}
\def\labelenumi{\arabic{enumi}.}
\tightlist
\item
  Calculer les distances de Haversine entre chaque paire d'entités.
\item
  Définir un seuil de distance maximale (\(d_{max}\)) :

  \begin{itemize}
  \tightlist
  \item
    Si \(d_{ij} < d_{max}\), \(w_{ij} = \frac{1}{d_{ij}}\);
  \item
    Sinon, \(w_{ij} = 0\).
  \end{itemize}
\item
  Normaliser les poids pour que chaque ligne de la matrice ait une somme
  égale à 1 : \[
   w_{ij}^{norm} = \frac{w_{ij}}{\sum_{j} w_{ij}}.
  \]
\end{enumerate}

\paragraph{Test significativité de l'indice de Moran
:}\label{test-significativituxe9-de-lindice-de-moran}

Le test de significativité de l'indice de Moran permet d'évaluer si une
variable présente une autocorrélation spatiale significative,
c'est-à-dire si les valeurs observées dans des zones proches ont
tendance à être similaires ou non.

\underline{\textbf{Hypothèses du test}}

\begin{itemize}
    \item \textbf{Hypothèse nulle} \( (H_0) \) : Il n’y a \textbf{pas d’autocorrélation spatiale} significative. Les valeurs observées sont distribuées de manière aléatoire dans l’espace.
    \item \textbf{Hypothèse alternative} \( (H_1) \) : Il existe une \textbf{autocorrélation spatiale} significative (positive ou négative).
\end{itemize}

\underline{\textbf{Statistique de test}}

L'Indice de Moran standardisé suit approximativement une distribution
normale sous \(H_0\). La statistique de test est donnée par :

\[
    Z = \frac{I - E(I)}{\text{Var}(I)}
\]

où :

\begin{itemize}
    \item \( I \) est l’Indice de Moran calculé sur les données;
    \item \( E(I) \) est l’espérance théorique de \( I \) sous \( H_0 \);
    \item \( \text{Var}(I) \) est la variance théorique de \( I \).
\end{itemize}

L'espérance sous \(H_0\) pour un échantillon de taille \(n\) est donnée
par :

\[
    E(I) = -\frac{1}{n - 1}
\]

\underline{\textbf{Règle de décision}}

On compare la statistique \(Z\) à une loi normale centrée réduite
\(\mathcal{N}(0,1)\). Pour un seuil de significativité \(\alpha\) (en
général 5 \%), on utilise les quantiles de la loi normale :

\begin{itemize}
    \item Si \( Z > z_{1-\alpha/2} \) ou \( Z < -z_{1-\alpha/2} \), on \textbf{rejette \( H_0 \)} et on conclut qu’il existe une autocorrélation spatiale significative.
    \item Si \( Z \in [-z_{1-\alpha/2}, z_{1-\alpha/2}] \), on \textbf{ne rejette pas \( H_0 \)}, et on considère que la distribution spatiale est aléatoire.
\end{itemize}

\underline{\textbf{Interprétation de l’Indice de Moran}}

\begin{itemize}
    \item \( I > 0 \) et significatif : Autocorrélation spatiale \textbf{positive} (les zones proches ont des valeurs similaires).
    \item \( I < 0 \) et significatif : Autocorrélation spatiale \textbf{négative} (les zones proches ont des valeurs éloignées).
    \item \( I \approx 0 \) significativement : Absence d’autocorrélation spatiale, les valeurs sont distribuées de manière aléatoire.
\end{itemize}

Ce test est couramment utilisé en analyse spatiale pour identifier des
regroupements de valeurs similaires, par exemple dans les études de
santé publique, d'aménagement du territoire ou d'économie régionale.

\subsection{Modélisation en économétrie
spatiale}\label{moduxe9lisation-en-uxe9conomuxe9trie-spatiale}

Voici un rappel des différents éléments utilisés dans l'ensemble des
modèles d'économétrie spatiale :

\begin{itemize}
\item[\(Y\)] : Il s'agit du vecteur des observations de la variable dépendante, c'est-à-dire la variable que l'on cherche à expliquer (par exemple, le taux de visites, le taux de chômage, etc).
\item[\(X\)] : C'est la matrice des variables explicatives ou indépendantes. Elle regroupe toutes les caractéristiques observées qui sont supposées influencer \textbf{Y} (comme des variables socio-économiques, démographiques ou structurelles).
\item[\(\beta\)] : Ce vecteur de coefficients mesure l'effet direct des variables \textbf{X} sur la variable dépendante \textbf{Y}. Chaque coefficient indique l'impact d'une unité de variation dans la variable correspondante sur \textbf{Y}, en l'absence d'effets spatiaux.
\item[\(W\)] : La matrice des poids spatiaux définit la structure de voisinage entre les unités géographiques. Chaque élément \(W_{ij}\) quantifie l'influence ou la proximité de l'unité \(j\) par rapport à l'unité \(i\). Le choix de cette matrice (par contiguïté, distance, ou K plus proches voisins) est crucial car il détermine la manière dont l'information spatiale est intégrée dans le modèle.
\item[\(WY\)] : Le terme de décalage spatial de \textbf{Y}, obtenu par le produit de la matrice \textbf{W} par le vecteur \textbf{Y}. Il représente l'influence moyenne pondérée des valeurs de \textbf{Y} dans les zones voisines et permet de capturer la dépendance spatiale directe de la variable dépendante.
\item[\(WX\)] : Il s'agit du terme de décalage spatial des variables explicatives. Concrètement, il représente une version pondérée des variables \textbf{X} dans les zones voisines, où les pondérations sont définies par la matrice \textbf{W}. Ce terme permet de mesurer l'effet indirect (ou spillover) des caractéristiques des voisins sur \textbf{Y}.
\item[\(\varepsilon\)] : C'est le terme d'erreur classique, qui capture les influences non observées ou aléatoires sur \textbf{Y}. Il est généralement supposé être indépendant et identiquement distribué (iid).
\item[\(\rho\)] : Utilisé dans les modèles qui intègrent directement l'effet des valeurs voisines de \textbf{Y} (comme dans les modèles SAR et SDM). Ce paramètre mesure la force de l'interaction entre la valeur de \textbf{Y} d'une unité et les valeurs de \textbf{Y} des unités voisines. Un \(\rho\) positif indique une autocorrélation positive (les zones avec des valeurs élevées de \textbf{Y} tendent à être entourées de zones à valeurs élevées, et inversement).
\item[\(\lambda\)] : Spécifique au modèle SEM (Spatial Error Model), ce paramètre quantifie la corrélation spatiale présente dans le terme d'erreur. Il mesure l'influence des erreurs des unités voisines sur l'erreur de l'unité considérée, suggérant que des facteurs non observés présentent une structure spatiale.
\item[\(\theta\)] : Ce vecteur de coefficients est associé au terme \textbf{WX} et apparaît dans les modèles SDM et SLX. Il mesure l'effet des variables explicatives des zones voisines sur la variable dépendante \textbf{Y}, c'est-à-dire l'impact indirect des caractéristiques locales via leur diffusion spatiale.
\end{itemize}

\subsubsection{Modèles principaux}\label{moduxe8les-principaux}

Le modèle général est défini comme suit :

\[
Y = \rho W Y + X \beta + \theta W X + u, \quad u = \lambda W u + \varepsilon
\]

\textbf{SAR (Spatial AutoRegressive Model)} :

Le modèle SAR introduit une dépendance spatiale directement sur la
variable dépendante \(Y\). L'idée est que la valeur de \(Y\) en un lieu
donné dépend des valeurs observées dans les zones voisines.
Mathématiquement, il s'écrit : \[
Y = \rho W Y + X \beta + \varepsilon
\]

\underline{\textbf{Interprétation :}}

\begin{itemize}
\item Si \( \rho > 0 \), les valeurs de \( Y \) ont tendance à être similaires entre voisins (autocorrélation positive).
\item Si \( \rho < 0 \), on observe un effet de dispersion, où les valeurs de \( Y \) sont opposées dans les zones voisines (autocorrélation négative).
\item Si \( \rho = 0 \), il n’y a pas de dépendance spatiale, et le modèle classique de régression linéaire est suffisant.
\end{itemize}

\textbf{SEM (Spatial Error Model)} :

Le modèle SEM est utilisé lorsque la dépendance spatiale affecte les
erreurs du modèle plutôt que la variable dépendante elle-même. Il est
défini par : \[
Y = X \beta + u, \quad u = \lambda W u + \varepsilon
\] \underline{\textbf{Interprétation :}}

\begin{itemize}
\item Contrairement au modèle SAR, le modèle SEM suppose que la dépendance spatiale est un effet de perturbation, provenant d’omissions de variables pertinentes qui suivent une structure spatiale.
\item Il est utilisé lorsque la corrélation spatiale détectée dans un modèle classique provient d’erreurs spatialement autocorrélées, plutôt que d’une interaction directe entre observations.
\end{itemize}

\textbf{SLX (Spatial Lag of X Model)} :

Le modèle SLX est plus facile à estimer, car il suppose que la variable
dépendante \(Y\) n'est pas directement influencée par les valeurs
voisines, mais uniquement par les variables explicatives des zones
voisines. Il est écrit comme suit : \[
Y = X \beta + \theta W X + \varepsilon
\] où \(W X\) capture l'effet des variables explicatives des unités
voisines.

\underline{\textbf{Interprétation :}}

\begin{itemize}
\item Il n’y a pas d’effet direct des valeurs voisines de \( Y \).
\item Il mesure uniquement l'effet de "spillover" (d'effet de débordement) des facteurs explicatifs.
\end{itemize}

\textbf{SDM (Spatial Durbin Model)} :

Le modèle SDM est une extension du modèle SAR. Il prend en compte non
seulement la dépendance de \(Y\) aux observations voisines, mais aussi
l'effet des variables explicatives des régions voisines. Il est défini
par : \[
Y = \rho W Y + X \beta + \theta W X + \varepsilon
\]

\underline{\textbf{Interprétation :}}

\begin{itemize}
\item Si \( \theta = 0 \), le modèle SDM devient un SAR classique.
\item Si \( \rho = 0 \), il devient un modèle SLX (voir ci-dessous).
\item Il permet de tester si des variables exogènes influencent \( Y \) au-delà des frontières administratives.
\end{itemize}

\textbf{Comparaison des modèles}

\begin{table}[h]
    \centering
    \begin{tabular}{|l|c|c|c|}
        \hline
        \textbf{Modèle} & \textbf{Dépendance spatiale sur \( Y \)} & \textbf{Effet des \( X \) des voisins} & \textbf{Effet des erreurs} \\
        \hline
        \textbf{SAR}  & Oui  & Non  & Non \\
        \hline
        \textbf{SEM}  & Non  & Non  & Oui \\
        \hline
        \textbf{SLX}  & Non  & Oui  & Non \\
        \hline
        \textbf{SDM}  & Oui  & Oui  & Non \\
        \hline
    \end{tabular}
    \caption{Comparaison des modèles spatiaux}
    \label{tab:comparaison_modeles}
\end{table}

\end{document}
