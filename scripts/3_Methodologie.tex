% Options for packages loaded elsewhere
\PassOptionsToPackage{unicode}{hyperref}
\PassOptionsToPackage{hyphens}{url}
%
\documentclass[
]{article}
\usepackage{amsmath,amssymb}
\usepackage{iftex}
\ifPDFTeX
  \usepackage[T1]{fontenc}
  \usepackage[utf8]{inputenc}
  \usepackage{textcomp} % provide euro and other symbols
\else % if luatex or xetex
  \usepackage{unicode-math} % this also loads fontspec
  \defaultfontfeatures{Scale=MatchLowercase}
  \defaultfontfeatures[\rmfamily]{Ligatures=TeX,Scale=1}
\fi
\usepackage{lmodern}
\ifPDFTeX\else
  % xetex/luatex font selection
\fi
% Use upquote if available, for straight quotes in verbatim environments
\IfFileExists{upquote.sty}{\usepackage{upquote}}{}
\IfFileExists{microtype.sty}{% use microtype if available
  \usepackage[]{microtype}
  \UseMicrotypeSet[protrusion]{basicmath} % disable protrusion for tt fonts
}{}
\makeatletter
\@ifundefined{KOMAClassName}{% if non-KOMA class
  \IfFileExists{parskip.sty}{%
    \usepackage{parskip}
  }{% else
    \setlength{\parindent}{0pt}
    \setlength{\parskip}{6pt plus 2pt minus 1pt}}
}{% if KOMA class
  \KOMAoptions{parskip=half}}
\makeatother
\usepackage{xcolor}
\usepackage[margin=1in]{geometry}
\usepackage{graphicx}
\makeatletter
\def\maxwidth{\ifdim\Gin@nat@width>\linewidth\linewidth\else\Gin@nat@width\fi}
\def\maxheight{\ifdim\Gin@nat@height>\textheight\textheight\else\Gin@nat@height\fi}
\makeatother
% Scale images if necessary, so that they will not overflow the page
% margins by default, and it is still possible to overwrite the defaults
% using explicit options in \includegraphics[width, height, ...]{}
\setkeys{Gin}{width=\maxwidth,height=\maxheight,keepaspectratio}
% Set default figure placement to htbp
\makeatletter
\def\fps@figure{htbp}
\makeatother
\setlength{\emergencystretch}{3em} % prevent overfull lines
\providecommand{\tightlist}{%
  \setlength{\itemsep}{0pt}\setlength{\parskip}{0pt}}
\setcounter{secnumdepth}{-\maxdimen} % remove section numbering
\ifLuaTeX
  \usepackage{selnolig}  % disable illegal ligatures
\fi
\usepackage{bookmark}
\IfFileExists{xurl.sty}{\usepackage{xurl}}{} % add URL line breaks if available
\urlstyle{same}
\hypersetup{
  hidelinks,
  pdfcreator={LaTeX via pandoc}}

\author{}
\date{\vspace{-2.5em}}

\begin{document}

\section{Méthodologie}\label{muxe9thodologie}

\subsection{Justification de l'économétrie
spatiale}\label{justification-de-luxe9conomuxe9trie-spatiale}

L'économétrie spatiale est justifiée par des raisons économiques et
économétriques. D'un point de vue économique, la proximité spatiale joue
un rôle clé dans les décisions des agents économiques. Les entreprises,
par exemple, ajustent leurs stratégies en fonction de la concurrence
locale, tandis que la diffusion des innovations et les effets
d'agglomération influencent la productivité régionale. Les externalités
spatiales, telles que l'effet de pair et les interactions entre
industries voisines, ont également un impact direct sur les marchés.
Ainsi, la prise en compte de la dimension spatiale est essentielle pour
comprendre les dynamiques économiques locales et globales.

Sur le plan économétrique, l'omission des effets spatiaux peut
introduire des biais dans les estimations, rendant les modèles
classiques inefficaces. L'autocorrélation spatiale des résidus (tout
comme l'autocorrélation temporelle des résidus) est un problème
récurrent, pouvant fausser l'inférence statistique si elle n'est pas
correctement modélisée. De plus, l'hypothèse d'indépendance des
observations, souvent supposée dans les modèles classiques, est rarement
vérifiée lorsque des interactions spatiales existent. En intégrant des
structures de dépendance spatiale, les modèles économétriques spatiaux
permettent d'améliorer la précision des estimations et de mieux
comprendre les relations entre unités géographiques, évitant ainsi les
erreurs d'interprétation liées à des phénomènes locaux ou régionaux.

\subsection{Concepts fondamentaux en statistique
spatiale}\label{concepts-fondamentaux-en-statistique-spatiale}

\subsubsection{Autocorrélation et hétérogénéité
spatiales}\label{autocorruxe9lation-et-huxe9tuxe9roguxe9nuxe9ituxe9-spatiales}

L'autocorrélation spatiale désigne la dépendance statistique entre des
observations géographiquement proches. En d'autres termes, les valeurs
prises par une variable en un lieu donné sont influencées par les
valeurs observées dans les localisations voisines. Cette dépendance peut
être positive, lorsque des valeurs similaires se regroupent, ou
négative, lorsqu'une valeur élevée en un point est associée à une valeur
faible dans les zones environnantes.

L'hétérogénéité spatiale, quant à elle, fait référence à la variabilité
des relations économiques en fonction de la localisation. Une même
variable explicative peut avoir des effets différents selon les régions.
Cette non-stationnarité spatiale implique qu'il est essentiel d'examiner
l'existence de régimes spatiaux distincts et d'adapter les modèles en
conséquence.

\subsubsection{Matrice des poids
spatiaux}\label{matrice-des-poids-spatiaux}

Pour quantifier la proximité spatiale entre unités géographiques, on
utilise une matrice de poids spatiaux notée \(W\). Cette matrice
représente les relations de voisinage et permet d'introduire la
structure spatiale dans les modèles économétriques. Il existe plusieurs
méthodes pour la définir :

\begin{itemize}
\item \textbf{Matrice de contiguïté} : Deux régions sont considérées comme voisines si elles partagent une frontière commune.
\item \textbf{Matrice de distance} : La pondération est inversement proportionnelle à la distance entre deux régions.
\item \textbf{Matrice des K plus proches voisins} : Chaque observation est associée aux $K$ unités les plus proches.
\end{itemize}

Une matrice de contiguïté binaire est définie comme suit : \[
W_{ij} = \begin{cases} 
1, & \text{si } i \text{ et } j \text{ sont voisins} \\
0, & \text{sinon} 
\end{cases}
\]

\subsubsection{Indices de corrélation
spatiale}\label{indices-de-corruxe9lation-spatiale}

L'un des indicateurs les plus couramment utilisés est l'indice de Moran,
défini par :

\[
I = \frac{N}{\sum_{i} \sum_{j} W_{ij}} \times \frac{\sum_{i} \sum_{j} W_{ij} (y_i - \bar{y})(y_j - \bar{y})}{\sum_{i} (y_i - \bar{y})^2}
\]

où \(y_i\) est la valeur de la variable d'intérêt en un point \(i\),
\(\bar{y}\) est la moyenne de cette variable et \(W_{ij}\) représente
l'élément \((i, j)\) de la matrice de poids.

D'autres indices existent, comme la \textbf{statistique de Geary}, qui
est moins sensible aux valeurs extrêmes, et les
\textbf{indicateurs locaux d'autocorrélation spatiale (LISA)}, qui
permettent d'identifier des clusters spatiaux spécifiques.

\subsection{Modélisation en économétrie
spatiale}\label{moduxe9lisation-en-uxe9conomuxe9trie-spatiale}

Voici un rappel des différents éléments utilisés dans l'ensemble des
modèles d'économétrie spatiale :

\begin{itemize}
\item[\(Y\)] : Il s'agit du vecteur des observations de la variable dépendante, c'est-à-dire la variable que l'on cherche à expliquer (par exemple, le taux de chômage, les prix immobiliers, etc.).
\item[\(X\)] : C'est la matrice des variables explicatives ou indépendantes. Elle regroupe toutes les caractéristiques observées qui sont supposées influencer \textbf{Y} (comme des variables socio-économiques, démographiques ou structurelles).
\item[\(\beta\)] : Ce vecteur de coefficients mesure l'effet direct des variables \textbf{X} sur la variable dépendante \textbf{Y}. Chaque coefficient indique l'impact d'une unité de variation dans la variable correspondante sur \textbf{Y}, en l'absence d'effets spatiaux.
\item[\(W\)] : La matrice des poids spatiaux définit la structure de voisinage entre les unités géographiques. Chaque élément \(W_{ij}\) quantifie l'influence ou la proximité de l'unité \(j\) par rapport à l'unité \(i\). Le choix de cette matrice (par contiguïté, distance, ou K plus proches voisins) est crucial car il détermine la manière dont l'information spatiale est intégrée dans le modèle.
\item[\(WY\)] Le terme de décalage spatial de \textbf{Y}, obtenu par le produit de la matrice \textbf{W} par le vecteur \textbf{Y}. Il représente l'influence moyenne pondérée des valeurs de \textbf{Y} dans les zones voisines et permet de capturer la dépendance spatiale directe de la variable dépendante.
\item[\(WX\)] : Il s'agit du terme de décalage spatial des variables explicatives. Concrètement, il représente une version pondérée des variables \textbf{X} dans les zones voisines, où les pondérations sont définies par la matrice \textbf{W}. Ce terme permet de mesurer l'effet indirect (ou spillover) des caractéristiques des voisins sur \textbf{Y}.
\item[\(\varepsilon\)] : C'est le terme d'erreur classique, qui capture les influences non observées ou aléatoires sur \textbf{Y}. Il est généralement supposé être indépendant et identiquement distribué (iid).
\item[\(\rho\)] : Utilisé dans les modèles qui intègrent directement l'effet des valeurs voisines de \textbf{Y} (comme dans les modèles SAR et SDM). Ce paramètre mesure la force de l'interaction entre la valeur de \textbf{Y} d'une unité et les valeurs de \textbf{Y} des unités voisines. Un \(\rho\) positif indique une autocorrélation positive (les zones avec des valeurs élevées de \textbf{Y} tendent à être entourées de zones à valeurs élevées, et inversement).
\item[\(\lambda\)] : Spécifique au modèle SEM (Spatial Error Model), ce paramètre quantifie la corrélation spatiale présente dans le terme d'erreur. Il mesure l'influence des erreurs des unités voisines sur l'erreur de l'unité considérée, suggérant que des facteurs non observés présentent une structure spatiale.
\item[\(\theta\)] : Ce vecteur de coefficients est associé au terme \textbf{WX} et apparaît dans les modèles SDM et SLX. Il mesure l'effet des variables explicatives des zones voisines sur la variable dépendante \textbf{Y}, c'est-à-dire l'impact indirect des caractéristiques locales via leur diffusion spatiale.
\end{itemize}

\subsubsection{Modèles principaux}\label{moduxe8les-principaux}

Le modèle général est défini comme suit :

\[
Y = \rho W Y + X \beta + \theta W X + u, \quad u = \lambda W u + \varepsilon
\]

\textbf{SAR (Spatial AutoRegressive Model)} : Le modèle SAR introduit
une dépendance spatiale directement sur la variable dépendante \(Y\).
L'idée est que la valeur de \(Y\) en un lieu donné dépend des valeurs
observées dans les zones voisines. Mathématiquement, il s'écrit : \[
Y = \rho W Y + X \beta + \varepsilon
\]

\underline{\textbf{Interprétation :}}

\begin{itemize}
\item Si \( \rho > 0 \), les valeurs de \( Y \) ont tendance à être similaires entre voisins (autocorrélation positive).
\item Si \( \rho < 0 \), on observe un effet de dispersion, où les valeurs de \( Y \) sont opposées dans les zones voisines (autocorrélation négative).
\item Si \( \rho = 0 \), il n’y a pas de dépendance spatiale, et le modèle classique de régression linéaire est suffisant.
\end{itemize}

\textbf{SEM (Spatial Error Model)} : Le modèle SEM est utilisé lorsque
la dépendance spatiale affecte les erreurs du modèle plutôt que la
variable dépendante elle-même. Il est défini par : \[
Y = X \beta + u, \quad u = \lambda W u + \varepsilon
\] \underline{\textbf{Interprétation :}}

\begin{itemize}
\item Contrairement au modèle SAR, le modèle SEM suppose que la dépendance spatiale est un effet de perturbation, provenant d’omissions de variables pertinentes qui suivent une structure spatiale.
\item Il est utilisé lorsque la corrélation spatiale détectée dans un modèle classique provient d’erreurs spatialement autocorrélées, plutôt que d’une interaction directe entre observations.
\end{itemize}

\textbf{SDM (Spatial Durbin Model)} : Le modèle SDM est une extension du
modèle SAR. Il prend en compte non seulement la dépendance de \(Y\) aux
observations voisines, mais aussi l'effet des variables explicatives des
régions voisines. Il est défini par : \[
Y = \rho W Y + X \beta + \theta W X + \varepsilon
\] \underline{\textbf{Interprétation :}}

\begin{itemize}
\item Si \( \theta = 0 \), le modèle SDM devient un SAR classique.
\item Si \( \rho = 0 \), il devient un modèle SLX (voir ci-dessous).
\item Il permet de tester si des variables exogènes influencent \( Y \) au-delà des frontières administratives.
\end{itemize}

\textbf{SLX (Spatial Lag of X Model)} : Le modèle SLX est plus simple
que SAR et SDM, car il suppose que la variable dépendante \(Y\) n'est
pas directement influencée par les valeurs voisines, mais uniquement par
les variables explicatives des zones voisines. Il est écrit comme suit :
\[
Y = X \beta + \theta W X + \varepsilon
\] où \(W X\) capture l'effet des variables explicatives des unités
voisines.

\underline{\textbf{Interprétation :}}

\begin{itemize}
\item Contrairement aux modèles SAR et SDM, il n’y a pas d’effet direct des valeurs voisines de \( Y \).
\item Il mesure uniquement l'effet de "spillover" (d'effet de débordement) des facteurs explicatifs.
\end{itemize}

\textbf{Comparaison des modèles}

\begin{table}[h]
    \centering
    \begin{tabular}{|l|c|c|c|}
        \hline
        \textbf{Modèle} & \textbf{Dépendance spatiale sur \( Y \)} & \textbf{Effet des \( X \) des voisins} & \textbf{Effet des erreurs} \\
        \hline
        \textbf{SAR}  & Oui  & Non  & Non \\
        \hline
        \textbf{SEM}  & Non  & Non  & Oui \\
        \hline
        \textbf{SDM}  & Oui  & Oui  & Non \\
        \hline
        \textbf{SLX}  & Non  & Oui  & Non \\
        \hline
    \end{tabular}
    \caption{Comparaison des modèles spatiaux}
    \label{tab:comparaison_modeles}
\end{table}

\subsection{Limites et difficultés}\label{limites-et-difficultuxe9s}

Le point sur les limites et difficultés en économétrie spatiale se
concentre sur plusieurs aspects. D'abord, la présence de données
manquantes pose un défi majeur, car les observations spatiales ne sont
pas toujours complètes. Pour y remédier, des méthodes comme l'imputation
par krigeage (une technique d'interpolation géostatistique) ou
l'estimation par maximum de vraisemblance adaptée aux données
incomplètes sont utilisées.

Ensuite, d'autres difficultés incluent le choix de la matrice de poids
spatiaux, qui influence fortement les résultats des modèles, et
l'hétérogénéité spatiale, qui peut nécessiter des techniques avancées
comme la régression géographiquement pondérée (GWR). Enfin, l'erreur
écologique et le problème du MAUP (Modifiable Areal Unit Problem)
compliquent l'interprétation des résultats, car les conclusions peuvent
varier selon le niveau d'agrégation des données.

\end{document}
