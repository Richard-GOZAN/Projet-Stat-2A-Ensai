% Options for packages loaded elsewhere
\PassOptionsToPackage{unicode}{hyperref}
\PassOptionsToPackage{hyphens}{url}
%
\documentclass[
]{article}
\usepackage{amsmath,amssymb}
\usepackage{iftex}
\ifPDFTeX
  \usepackage[T1]{fontenc}
  \usepackage[utf8]{inputenc}
  \usepackage{textcomp} % provide euro and other symbols
\else % if luatex or xetex
  \usepackage{unicode-math} % this also loads fontspec
  \defaultfontfeatures{Scale=MatchLowercase}
  \defaultfontfeatures[\rmfamily]{Ligatures=TeX,Scale=1}
\fi
\usepackage{lmodern}
\ifPDFTeX\else
  % xetex/luatex font selection
\fi
% Use upquote if available, for straight quotes in verbatim environments
\IfFileExists{upquote.sty}{\usepackage{upquote}}{}
\IfFileExists{microtype.sty}{% use microtype if available
  \usepackage[]{microtype}
  \UseMicrotypeSet[protrusion]{basicmath} % disable protrusion for tt fonts
}{}
\makeatletter
\@ifundefined{KOMAClassName}{% if non-KOMA class
  \IfFileExists{parskip.sty}{%
    \usepackage{parskip}
  }{% else
    \setlength{\parindent}{0pt}
    \setlength{\parskip}{6pt plus 2pt minus 1pt}}
}{% if KOMA class
  \KOMAoptions{parskip=half}}
\makeatother
\usepackage{xcolor}
\usepackage[margin=1in]{geometry}
\usepackage{graphicx}
\makeatletter
\def\maxwidth{\ifdim\Gin@nat@width>\linewidth\linewidth\else\Gin@nat@width\fi}
\def\maxheight{\ifdim\Gin@nat@height>\textheight\textheight\else\Gin@nat@height\fi}
\makeatother
% Scale images if necessary, so that they will not overflow the page
% margins by default, and it is still possible to overwrite the defaults
% using explicit options in \includegraphics[width, height, ...]{}
\setkeys{Gin}{width=\maxwidth,height=\maxheight,keepaspectratio}
% Set default figure placement to htbp
\makeatletter
\def\fps@figure{htbp}
\makeatother
\setlength{\emergencystretch}{3em} % prevent overfull lines
\providecommand{\tightlist}{%
  \setlength{\itemsep}{0pt}\setlength{\parskip}{0pt}}
\setcounter{secnumdepth}{-\maxdimen} % remove section numbering
\ifLuaTeX
  \usepackage{selnolig}  % disable illegal ligatures
\fi
\usepackage{bookmark}
\IfFileExists{xurl.sty}{\usepackage{xurl}}{} % add URL line breaks if available
\urlstyle{same}
\hypersetup{
  hidelinks,
  pdfcreator={LaTeX via pandoc}}

\author{}
\date{\vspace{-2.5em}}

\begin{document}

\section{Méthodologie}\label{muxe9thodologie}

\subsection{\texorpdfstring{Définition de l'Effet Aléatoire
\(v_i\)}{Définition de l'Effet Aléatoire v\_i}}\label{duxe9finition-de-leffet-aluxe9atoire-v_i}

L'effet aléatoire \(v_i\) capture l'effet d'hétérogénéité non mesurable
qui influence les résultats dans chaque agglomération. Il peut être
considéré comme une variation spécifique à l'agglomération qui n'est pas
expliquée par les covariables observées dans le modèle (comme les
facteurs socio-économiques).

\subsection{\texorpdfstring{Estimation de l'Effet Aléatoire \(v_i\) dans
les modèles
mixtes}{Estimation de l'Effet Aléatoire v\_i dans les modèles mixtes}}\label{estimation-de-leffet-aluxe9atoire-v_i-dans-les-moduxe8les-mixtes}

Pour une agglomération donnée \(i\), l'effet aléatoire \(v_i\)
représente la composante non observée qui affecte le nombre attendu de
cas d'un événement (par exemple, le cancer œsophagien) dans cette
agglomération. Voici comment cet effet aléatoire est généralement défini
et estimé.

\subsubsection{Structure de Variance}\label{structure-de-variance}

Dans le cadre d'un modèle mixte, l'effet \(v_i\) est généralement
supposé suivre une distribution normale multivariée :

\[
v_i \sim N(0, \sigma_v^2)
\]

où \(\sigma_v^2\) est la variance des effets aléatoires. Cela signifie
que la plupart des valeurs de \(v_i\) seront proches de zéro, certains
pouvant être positifs (indiquant un risque plus élevé que prévu) et
d'autres négatifs (indiquant un risque plus faible).

\subsubsection{\texorpdfstring{Estimation de
\(v_i\)}{Estimation de v\_i}}\label{estimation-de-v_i}

\subsection{Méthodes d'Estimation}\label{muxe9thodes-destimation}

Pour une agglomération donnée \(i\), l'estimation de \(v_i\) peut être
réalisée à l'aide des méthodes suivantes :

\subsubsection{Approche de Maximum de Vraisemblance
(ML)}\label{approche-de-maximum-de-vraisemblance-ml}

\begin{itemize}
\item
  Une fois que le modèle est ajusté, les effets aléatoires sont souvent
  estimés comme partie intégrante du processus d'optimisation de la
  vraisemblance. Cela se fait généralement en maximisant la
  vraisemblance marginale des données en tenant compte des effets
  aléatoires.
\item
  Après avoir obtenu les estimations des paramètres du modèle, on peut
  dériver les estimations de \(v_i\) à partir des résidus du modèle.
\end{itemize}

\subsubsection{Approche Empirique}\label{approche-empirique}

\begin{itemize}
\tightlist
\item
  Après l'ajustement du modèle, les valeurs de \(v_i\) peuvent être
  calculées à partir des résidus du modèle pour chaque agglomération.
  Cela implique de comparer les cas observés dans chaque agglomération à
  ceux prédits par le modèle, en tenant compte de l'effet des
  covariables.
\end{itemize}

\subsection{Utilisation des Effets
Aléatoires}\label{utilisation-des-effets-aluxe9atoires}

L'effet \(v_i\) est essentiel pour comprendre comment des facteurs non
mesurés ou non observés influencent les résultats dans une agglomération
\(i\). Il permet aussi :

\begin{itemize}
\tightlist
\item
  D'ajuster pour la variabilité non expliquée par les facteurs observés.
\item
  D'améliorer la précision des estimations de risque dans le modèle en
  tenant compte des corrélations spatiales.
\end{itemize}

En résumé, pour une agglomération donnée \(i\), \(v_i\) représente le
composant aléatoire du modèle qui ajuste l'incidence observée à celle
attendue, en tenant compte de la structure de dépendance spatiale et des
effets non mesurés. L'estimation de \(v_i\) se fait généralement par des
méthodes comme le maximum de vraisemblance ou l'analyse des résidus
après ajustement du modèle.

\subsection{\texorpdfstring{Effets Aléatoires Spatiaux Basés sur le
Voisinage
\(w_i\)}{Effets Aléatoires Spatiaux Basés sur le Voisinage w\_i}}\label{effets-aluxe9atoires-spatiaux-basuxe9s-sur-le-voisinage-w_i}

\subsubsection{Modèle de Voisinage}\label{moduxe8le-de-voisinage}

Dans le modèle avec effets aléatoires basés sur le voisinage, \(w_i\)
représente la variation qui dépend des valeurs observées dans les zones
voisines. Ceci est souvent formulé en utilisant un \textbf{modèle
autoregressif conditionnel (CAR)}.

\subsubsection{Structure des Modèles
CAR}\label{structure-des-moduxe8les-car}

Le modèle CAR peut être spécifié comme suit :

\[
w_i \mid w_j, j \neq i \sim N \left( \frac{1}{n_i} \sum_{j \in \delta_i} w_j, \sigma^2 \right)
\]

où :

\begin{itemize}
\tightlist
\item
  \(n_i\) est le nombre de voisins pour l'agglomération \(i\),
\item
  \(\delta_i\) est l'ensemble des voisins de l'agglomération \(i\),
\item
  \(\sigma^2\) est la variance de l'effet aléatoire.
\end{itemize}

Ce modèle suppose que la valeur de \(w_i\) dépend en moyenne des valeurs
observées dans les zones voisines, ce qui permet de modéliser la
corrélation spatiale.

\subsubsection{Estimation des
Paramètres}\label{estimation-des-paramuxe8tres}

Pour estimer les paramètres de ces modèles (y compris \(v_i\) et
\(w_i\)), une approche courante consiste à utiliser des méthodes
spécifiques dans le cadre de la \textbf{maximum de vraisemblance} ou des
\textbf{approches bayésiennes}.

\subsubsection{Étapes d'Estimation}\label{uxe9tapes-destimation}

\begin{enumerate}
\def\labelenumi{\arabic{enumi}.}
\tightlist
\item
  \textbf{Pseudo-vraisemblance}

  \begin{itemize}
  \tightlist
  \item
    On utilise la vraisemblance conditionnelle de \(Y\) étant donné les
    effets aléatoires, pour améliorer l'estimation des paramètres.
  \item
    Cette approche permet de simplifier l'estimation en contournant
    certaines intégrations complexes dans les modèles à effets
    aléatoires.
  \end{itemize}
\item
  \textbf{Optimisation}

  \begin{itemize}
  \tightlist
  \item
    Les algorithmes d'optimisation, tels que \textbf{Newton-Raphson},
    sont utilisés pour maximiser la vraisemblance.
  \item
    Pour des approches bayésiennes, des méthodes de simulation comme
    \textbf{MCMC (Markov Chain Monte Carlo)} sont couramment appliquées
    pour estimer \(v_i\) et \(w_i\).
  \end{itemize}
\end{enumerate}

\subsection{Modèle de régression
logistique}\label{moduxe8le-de-ruxe9gression-logistique}

Elle modélise une réponse binaire (\(y \sim B(n, p)\)), où \(p\) est la
probabilité de succès : \[
P(y | n, p) = \binom{n}{y} p^y (1-p)^{n-y}
\] La probabilité \(p\) est reliée au prédicteur par la fonction
logistique :

\[
p = \frac{1}{1 + e^{-\eta}} \quad \text{où} \quad \eta = \beta_0 + \sum_{i=1}^m \beta_i x_i.
\] La log-vraisemblance est exprimée comme :

\[
\ell(\boldsymbol{\beta}) = \sum_{i=1}^n \left[ y_i \log{p_i} + (1-y_i) \log{(1-p_i)} \right].
\]

\subsection{Modèle de régression de
Poisson}\label{moduxe8le-de-ruxe9gression-de-poisson}

La régression de Poisson est un type de modèle linéaire généralisé (GLM)
utilisé pour modéliser des données de comptage. La distribution de
Poisson suppose que la variance est égale à la moyenne, ce qui est
souvent approprié pour des données de comptage. Le modèle de base est :

\[
\log(\mu_i) = \beta_0 + \beta_1 X_{i1} + \beta_2 X_{i2} + \dots + \beta_p X_{ip} = (X\beta)_i
\]

où \(\mu_i\) est le nombre attendu de cas pour l'observation \(i\),
\(X_{ij}\) sont les variables explicatives, et \(\beta_j\) sont les
coefficients à estimer.

\subsection{Modèles de Régression de Poisson avec Effets
Aléatoires}\label{moduxe8les-de-ruxe9gression-de-poisson-avec-effets-aluxe9atoires}

\subsubsection{Régression Poisson avec Effets Aléatoires Non
Spatiaux}\label{ruxe9gression-poisson-avec-effets-aluxe9atoires-non-spatiaux}

La formulation générale de ce modèle peut être écrite comme suit :

\[
\log(\mu_i) = \log(E_i) + (X\beta)_i + u_i
\]

où :

\begin{itemize}
\tightlist
\item
  \(\mu_i\) est la moyenne du nombre de visites pour l'agglomération
  \(i\),
\item
  \(E_i\) est la taille de la population de l'agglomération \(i\) (terme
  d'ajustement de l'exposition),
\item
  \(X\) est la matrice de conception pour les facteurs
  socio-économiques,
\item
  \(\beta\) est le vecteur des coefficients à estimer,
\item
  \(u_i\) est l'effet aléatoire spécifique à l'agglomération \(i\),
  supposé indépendant et suivant une distribution normale.
\end{itemize}

\subsubsection{Régression Poisson avec effets aléatoires spatiaux basés
sur la
Distance}\label{ruxe9gression-poisson-avec-effets-aluxe9atoires-spatiaux-basuxe9s-sur-la-distance}

Ce modèle prend en compte l'autocorrélation spatiale par l'introduction
d'effets aléatoires spatiaux :

\[
\log(\mu_i) = \log(E_i) + (X\beta)_i + u_i + v_i
\]

où :

\begin{itemize}
\tightlist
\item
  \(v_i\) est l'effet aléatoire spatial basé sur la distance, qui peut
  suivre un \textbf{processus gaussien} avec une certaine structure de
  covariance (définie plus haut).
\end{itemize}

\subsubsection{Régression Poisson avec effets aléatoires spatiaux basés
sur le
voisinage}\label{ruxe9gression-poisson-avec-effets-aluxe9atoires-spatiaux-basuxe9s-sur-le-voisinage}

Ce modèle est similaire au précédent, mais \(v_i\) est remplacé par un
effet aléatoire basé sur le voisinage :

\[
\log(\mu_i) = \log(E_i) + (X\beta)_i + u_i + w_i
\]

où :

\begin{itemize}
\tightlist
\item
  \(w_i\) représente l'effet aléatoire spatial basé sur le voisinage
  (par exemple, un \textbf{modèle autoregressif conditionnel (CAR)} qui
  capture l'influence des agglomérations voisines).
\end{itemize}

\subsection{Note sur l'Estimation}\label{note-sur-lestimation}

Pour tous ces modèles, l'estimation des paramètres est réalisée à l'aide
de la \textbf{méthode de pseudo-vraisemblance}, qui prend en compte les
structures autocorrélées.

De plus, pour la comparaison des modèles, on utilise les critères
suivants :

\begin{itemize}
\tightlist
\item
  \textbf{AIC (Akaike Information Criterion)} : évalue la qualité du
  modèle en pénalisant les modèles trop complexes.
\item
  \textbf{BIC (Bayesian Information Criterion)} : similaire à l'AIC,
  mais avec une pénalisation plus forte sur la complexité du modèle.
\end{itemize}

Ces critères permettent de choisir le modèle le plus adapté en fonction
des données disponibles.

\end{document}
