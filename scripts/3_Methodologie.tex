% Options for packages loaded elsewhere
\PassOptionsToPackage{unicode}{hyperref}
\PassOptionsToPackage{hyphens}{url}
%
\documentclass[
]{article}
\usepackage{amsmath,amssymb}
\usepackage{iftex}
\ifPDFTeX
  \usepackage[T1]{fontenc}
  \usepackage[utf8]{inputenc}
  \usepackage{textcomp} % provide euro and other symbols
\else % if luatex or xetex
  \usepackage{unicode-math} % this also loads fontspec
  \defaultfontfeatures{Scale=MatchLowercase}
  \defaultfontfeatures[\rmfamily]{Ligatures=TeX,Scale=1}
\fi
\usepackage{lmodern}
\ifPDFTeX\else
  % xetex/luatex font selection
\fi
% Use upquote if available, for straight quotes in verbatim environments
\IfFileExists{upquote.sty}{\usepackage{upquote}}{}
\IfFileExists{microtype.sty}{% use microtype if available
  \usepackage[]{microtype}
  \UseMicrotypeSet[protrusion]{basicmath} % disable protrusion for tt fonts
}{}
\makeatletter
\@ifundefined{KOMAClassName}{% if non-KOMA class
  \IfFileExists{parskip.sty}{%
    \usepackage{parskip}
  }{% else
    \setlength{\parindent}{0pt}
    \setlength{\parskip}{6pt plus 2pt minus 1pt}}
}{% if KOMA class
  \KOMAoptions{parskip=half}}
\makeatother
\usepackage{xcolor}
\usepackage[margin=1in]{geometry}
\usepackage{graphicx}
\makeatletter
\def\maxwidth{\ifdim\Gin@nat@width>\linewidth\linewidth\else\Gin@nat@width\fi}
\def\maxheight{\ifdim\Gin@nat@height>\textheight\textheight\else\Gin@nat@height\fi}
\makeatother
% Scale images if necessary, so that they will not overflow the page
% margins by default, and it is still possible to overwrite the defaults
% using explicit options in \includegraphics[width, height, ...]{}
\setkeys{Gin}{width=\maxwidth,height=\maxheight,keepaspectratio}
% Set default figure placement to htbp
\makeatletter
\def\fps@figure{htbp}
\makeatother
\setlength{\emergencystretch}{3em} % prevent overfull lines
\providecommand{\tightlist}{%
  \setlength{\itemsep}{0pt}\setlength{\parskip}{0pt}}
\setcounter{secnumdepth}{-\maxdimen} % remove section numbering
\ifLuaTeX
  \usepackage{selnolig}  % disable illegal ligatures
\fi
\usepackage{bookmark}
\IfFileExists{xurl.sty}{\usepackage{xurl}}{} % add URL line breaks if available
\urlstyle{same}
\hypersetup{
  hidelinks,
  pdfcreator={LaTeX via pandoc}}

\author{}
\date{\vspace{-2.5em}}

\begin{document}

\section{Méthodologie}\label{muxe9thodologie}

Dans le cadre de cette étude, nous cherchons à modéliser le nombre de
consultations médicales en prenant en compte les caractéristiques
socio-économiques et spatiales des communes. La littérature suggère que
les modèles de régression de Poisson et ses variantes avec effets
aléatoires sont particulièrement adaptés aux données de comptage, tout
en tenant compte de l'hétérogénéité locale et des dépendances spatiales.
Ainsi, nous proposons une approche combinant les modèles de régression
de Poisson standard et les modèles mixtes généralisés pour mieux
comprendre les dynamiques d'accès aux soins.

\subsection{Modélisation du nombre de visites de médécins
généralistes}\label{moduxe9lisation-du-nombre-de-visites-de-muxe9duxe9cins-guxe9nuxe9ralistes}

\subsubsection{Modèle de régression de
Poisson}\label{moduxe8le-de-ruxe9gression-de-poisson}

La régression de Poisson est un type de modèle linéaire généralisé (GLM)
utilisé pour modéliser des données de comptage. La distribution de
Poisson suppose que la variance est égale à la moyenne, ce qui est
souvent approprié pour des données de comptage. Le modèle de base est :

\[
\log(\mu_i) = \beta_0 + \beta_1 X_{i1} + \beta_2 X_{i2} + \dots + \beta_p X_{ip} = (X\beta)_i
\]

où \(\mu_i\) est le nombre attendu de cas pour l'observation \(i\),
\(X_{ij}\) sont les variables explicatives, et \(\beta_j\) sont les
coefficients à estimer. Toutefois, ce modèle suppose une indépendance
entre observations, ce qui peut ne pas être réaliste en présence de
corrélation spatiale ou d'hétérogénéité non observée.

\subsubsection{Modèles de Régression de Poisson avec effets
aléatoires}\label{moduxe8les-de-ruxe9gression-de-poisson-avec-effets-aluxe9atoires}

Afin de mieux capter l'hétérogénéité entre les communes, nous
introduisons des modèles avec effets aléatoires. Ces modèles permettent
de prendre en compte des facteurs non observés qui influencent le nombre
de consultations.

Un effet aléatoire \(v_i\) capture l'effet d'hétérogénéité non mesurable
qui influence les résultats dans chaque agglomération. Il peut être
considéré comme une variation spécifique à l'agglomération qui n'est pas
expliquée par les covariables observées dans le modèle (comme les
facteurs socio-économiques).

\paragraph{Régression de Poisson avec effets aléatoires non
spatiaux}\label{ruxe9gression-de-poisson-avec-effets-aluxe9atoires-non-spatiaux}

La formulation générale de ce modèle peut être écrite comme suit :

\[
\log(\mu_i) = \log(E_i) + (X\beta)_i + u_i
\]

où :

\begin{itemize}
\tightlist
\item
  \(\mu_i\) est la moyenne du nombre de visites pour l'agglomération
  \(i\),
\item
  \(E_i\) est la taille de la population de l'agglomération \(i\) (terme
  d'ajustement de l'exposition),
\item
  \(X\) est la matrice de conception pour les facteurs
  socio-économiques,
\item
  \(\beta\) est le vecteur des coefficients à estimer,
\item
  \(u_i\) est l'effet aléatoire spécifique à l'agglomération \(i\),
  supposé indépendant et suivant une distribution normale.
\end{itemize}

\paragraph{Régression Poisson avec effets aléatoires
spatiaux}\label{ruxe9gression-poisson-avec-effets-aluxe9atoires-spatiaux}

Deux approches sont utilisées pour modéliser la structure spatiale des
données :

\begin{itemize}
\item
  Effets aléatoires basés sur la distance : ajout d'un effet spatial
  \(v_i\) capturant la corrélation entre communes distantes.
\item
  Effets aléatoires basés sur le voisinage : utilisation d'un terme
  \(w_i\) représentant les influences des communes voisines à travers un
  modèle autorégressif conditionnel (CAR).
\end{itemize}

Ce modèle prend en compte l'autocorrélation spatiale par l'introduction
d'effets aléatoires spatiaux :

\[
\log(\mu_i) = \log(E_i) + (X\beta)_i + u_i + w_i
\] où \(w_i\) suit une distribution conditionnelle dépendant des
voisins.

Le modèle CAR peut être spécifié comme suit :

\[
w_i \mid w_j, j \neq i \sim N \left( \frac{1}{n_i} \sum_{j \in \delta_i} w_j, \sigma^2 \right)
\]

où :

\begin{itemize}
\tightlist
\item
  \(n_i\) est le nombre de voisins pour l'agglomération \(i\),
\item
  \(\delta_i\) est l'ensemble des voisins de l'agglomération \(i\),
\item
  \(\sigma^2\) est la variance de l'effet aléatoire.
\end{itemize}

Ce modèle suppose que la valeur de \(w_i\) dépend en moyenne des valeurs
observées dans les zones voisines, ce qui permet de modéliser la
corrélation spatiale.

\subsection{Estimation des
paramètres}\label{estimation-des-paramuxe8tres}

\subsubsection{Structure de variance des effets
aléatoires}\label{structure-de-variance-des-effets-aluxe9atoires}

Dans le cadre d'un modèle mixte, les effets aléatoires \(v_i\) et
\(w_i\) sont supposés suivre une distribution normale centrée avec une
variance \(\sigma_v^2\) et \(\sigma_w^2\) respectivement.

\subsubsection{Méthodes d'estimation}\label{muxe9thodes-destimation}

Deux principales méthodes sont utilisées pour estimer les paramètres des
modèles

\emph{Approche de Maximum de Vraisemblance (ML)}

Une fois que le modèle est ajusté, les effets aléatoires sont souvent
estimés comme partie intégrante du processus d'optimisation de la
vraisemblance. Cela se fait généralement en maximisant la vraisemblance
marginale des données en tenant compte des effets aléatoires.

Après avoir obtenu les estimations des paramètres du modèle, on peut
dériver les estimations de \(v_i\) à partir des résidus du modèle.

\emph{Approche Empirique}

Après l'ajustement du modèle, les valeurs de \(v_i\) peuvent être
calculées à partir des résidus du modèle pour chaque agglomération. Cela
implique de comparer les cas observés dans chaque agglomération à ceux
prédits par le modèle, en tenant compte de l'effet des covariables.

\subsubsection{Étapes d'Estimation}\label{uxe9tapes-destimation}

Pour estimer les paramètres de ces modèles (y compris \(v_i\) et
\(w_i\)), une approche courante consiste à utiliser des méthodes
spécifiques dans le cadre de la \textbf{maximum de vraisemblance} ou des
\textbf{approches bayésiennes}.

\begin{enumerate}
\def\labelenumi{\arabic{enumi}.}
\tightlist
\item
  \textbf{Pseudo-vraisemblance}

  \begin{itemize}
  \tightlist
  \item
    On utilise la vraisemblance conditionnelle de \(Y\) étant donné les
    effets aléatoires, pour améliorer l'estimation des paramètres.
  \item
    Cette approche permet de simplifier l'estimation en contournant
    certaines intégrations complexes dans les modèles à effets
    aléatoires.
  \end{itemize}
\item
  \textbf{Optimisation}

  \begin{itemize}
  \tightlist
  \item
    Les algorithmes d'optimisation, tels que \textbf{Newton-Raphson},
    sont utilisés pour maximiser la vraisemblance.
  \item
    Pour des approches bayésiennes, des méthodes de simulation comme
    \textbf{MCMC (Markov Chain Monte Carlo)} sont couramment appliquées
    pour estimer \(v_i\) et \(w_i\).
  \end{itemize}
\end{enumerate}

Ce modèle est similaire au précédent, mais \(v_i\) est remplacé par un
effet aléatoire basé sur le voisinage :

\[
\log(\mu_i) = \log(E_i) + (X\beta)_i + u_i + w_i
\]

où :

\begin{itemize}
\tightlist
\item
  \(w_i\) représente l'effet aléatoire spatial basé sur le voisinage
  (par exemple, un \textbf{modèle autoregressif conditionnel (CAR)} qui
  capture l'influence des agglomérations voisines).
\end{itemize}

\subsection{Modèle de régression
logistique}\label{moduxe8le-de-ruxe9gression-logistique}

Elle modélise une réponse binaire (\(y \sim B(n, p)\)), où \(p\) est la
probabilité de succès : \[
P(y | n, p) = \binom{n}{y} p^y (1-p)^{n-y}
\] La probabilité \(p\) est reliée au prédicteur par la fonction
logistique :

\[
p = \frac{1}{1 + e^{-\eta}} \quad \text{où} \quad \eta = \beta_0 + \sum_{i=1}^m \beta_i x_i.
\] La log-vraisemblance est exprimée comme :

\[
\ell(\boldsymbol{\beta}) = \sum_{i=1}^n \left[ y_i \log{p_i} + (1-y_i) \log{(1-p_i)} \right].
\]

\subsection{Note sur l'Estimation}\label{note-sur-lestimation}

Pour tous ces modèles, l'estimation des paramètres est réalisée à l'aide
de la \textbf{méthode de pseudo-vraisemblance}, qui prend en compte les
structures autocorrélées.

De plus, pour la comparaison des modèles, on utilise les critères
suivants :

\begin{itemize}
\tightlist
\item
  \textbf{AIC (Akaike Information Criterion)} : évalue la qualité du
  modèle en pénalisant les modèles trop complexes.
\item
  \textbf{BIC (Bayesian Information Criterion)} : similaire à l'AIC,
  mais avec une pénalisation plus forte sur la complexité du modèle.
\end{itemize}

Ces critères permettent de choisir le modèle le plus adapté en fonction
des données disponibles.

Cette méthodologie propose une modélisation rigoureuse du nombre de
consultations en combinant régression de Poisson et modèles mixtes.
L'ajout d'effets aléatoires et de structures spatiales permet
d'améliorer la précision des estimations et d'obtenir une meilleure
compréhension des disparités géographiques dans l'accès aux soins. La
comparaison des modèles permettra d'identifier la structure la plus
adaptée aux données étudiées.

\end{document}
