% === Chargement des packages ===
\usepackage{amsmath, amssymb, geometry, pdfpages, graphicx, atbegshi, fancyhdr, tocloft, tcolorbox, xcolor, titlesec}

% === Définition des couleurs ===
\definecolor{bleu}{RGB}{0, 50, 150}      % Bleu foncé
\definecolor{marron}{RGB}{139, 69, 19}   % Marron
\definecolor{gris}{RGB}{100, 100, 100}   % Gris
\definecolor{cadre}{RGB}{220, 220, 220}  % Gris clair pour encadrer les équations

% === Configuration de la géométrie du document ===
\geometry{a4paper, margin=2.5cm}

% === Personnalisation des en-têtes et pieds de page ===
\pagestyle{fancy}
\fancyhf{}
\renewcommand{\headrulewidth}{0.4pt}
\renewcommand{\footrulewidth}{0.4pt}
\fancyhead[L]{\textbf{\textcolor{bleu}{Élèves Ingénieurs}}}
\fancyhead[R]{\textbf{\textcolor{marron}{@Alex, Ali, Richard \& Toussaint}}}
\fancyfoot[L]{\textbf{Mars 2025}}
\fancyfoot[C]{\thepage}
\fancyfoot[R]{\textbf{\textcolor{gris}{Projet Statistique}}}

% === Table des matières stylisée ===
\setcounter{tocdepth}{5}                
\renewcommand\contentsname{\begin{center}\textcolor{marron}{\Huge\textbf{Sommaire}}\end{center}}

% === Appliquer les en-têtes et pieds de page dès la première page ===
\AtBeginDocument{
  \thispagestyle{fancy}
}

% === Personnalisation des titres ===
% TITRE DE CHAPITRE : Bleu, centré, numéroté
\titleformat{\chapter}
  {\centering\normalfont\Huge\bfseries\color{bleu}}{\thechapter}{1em}{}

% TITRE DE SECTION : Marron, numéroté
\titleformat{\section}
  {\normalfont\LARGE\bfseries\color{marron}}{\thesection}{1em}{}

% TITRE DE SOUS-SECTION : Gris, numéroté
\titleformat{\subsection}
  {\normalfont\Large\bfseries\color{gris}}{\thesubsection}{1em}{}

% TITRE DE SOUS-SOUS-SECTION : Noir, en gras, numéroté
\titleformat{\subsubsection}
  {\normalfont\bfseries\large\color{black}}{\thesubsubsection}{1em}{}

% TITRE DE PARAGRAPHES : Italique, en noir, numéroté
\titleformat{\paragraph}
  {\normalfont\itshape\normalsize\color{black}}{\theparagraph}{1em}{}

% TITRE DE SOUS-PARAGRAPHES : Italique, en gris foncé, numéroté
\titleformat{\subparagraph}
  {\normalfont\itshape\small\color{gray}}{\thesubparagraph}{1em}{}


% === Encadrement des équations ===
\usepackage{empheq}
\newtcolorbox{eqbox}{
  colframe=cadre,
  colback=white,
  sharp corners,
  boxrule=1pt,
  left=10pt,
  right=10pt,
  top=5pt,
  bottom=5pt
}

% Redéfinition de l'environnement equation avec encadrement
\renewenvironment{equation}{
  \begin{eqbox}
  \begin{equation*}
}{
  \end{equation*}
  \end{eqbox}
}
